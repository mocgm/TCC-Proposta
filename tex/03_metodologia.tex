\chapter{Metodologia}\label{cap:metodologia}

A metodologia é uma descrição técnica de como será desenvolvido ou foi desenvolvido o trabalho. Devem estar detalhadas, de forma lógica, linear e cronológica, todas as etapas do projeto.

Deve ser explicado como o produto será gerado, quais são os principais fundamentos (algoritmos, paradigmas, teorias) e tecnologias (ambientes de desenvolvimento, linguagens de programação, plataformas de hardware) a serem empregados.

\section{Fundamentos}\label{s:fundamentos}
Indicar quais são os quais são os principais fundamentos (algoritmos, paradigmas, teorias) a serem empregados.
Cada fundamento utilizado deve ser justificado.

\section{Tecnologias}\label{s:tecnologias}
Indicar quais são os quais são as principais tecnologias (ambientes de desenvolvimento, linguagens de programação, plataformas de hardware) a serem empregadas. Cada tecnologia utilizada deve ser justificada.

\section{Como utilizar figuras}\label{s:usoFiguras}

A legenda de uma figura deve estar abaixo da mesma (ver Figura \ref{fig:ElemPreTextuais}).

\begin{figure}[h]
\begin{center}
    \includegraphics[width=0.5\columnwidth]{./ElemPreTextuais.png}
    \caption{Elementos Pré-Textuais de um Trabalho Acadêmico.}Fonte: \cite{NormasUTFPR}
    \label{fig:ElemPreTextuais}
\end{center}
\end{figure}

\section{Como utilizar tabelas}\label{s:usoTabelas}

O título de uma tabela deve estar acima da mesma. A Tabela \ref{tab:publicacao} ilustra um exemplo de uma tabela criada no latex\footnote{Se é a primeira vez que está trabalhando com o latex, comece acessando a página do Prof. Hugo Vieira em http://pessoal.utfpr.edu.br/hvieir/orient/}.

\begin{table}[h!]
%Se o nome da tabela for muito grande, é possível colocar entre colchetes o título reduzido que será mostrado na lista de tabelas e entre aspas o título completo que será mostrado acima da tabela. O mesmo é válido para figuras.

\caption[Distribuição do percentual de artigos segundo o ano de publicação]{Distribuição do percentual de artigos segundo o ano de publicação, período entre 1995 a 2005.}
\begin{center}
%\scalebox{0.75}{
\begin{tabular}{|c|c|c|c|c|c|}
\hline
\textbf{Ano de Publicação} &\textbf{Número de Artigos} & \textbf{\simbolo{\%}{Percentual} de Artigos} \\ \hline \hline
1995 & 1 & 2,6	\\
1998 & 3 &	7,9	\\
1999 & 3 &	7,9  \\
2000 & 2 & 5,2 \\
2001 & 5 & 13,2 \\
2002 & 5 &	13,2 \\
2003 & 5 &	13,2  \\
2004 & 7 &	18,4 \\
2005 & 7 & 18,4 \\
Total & 38 & 100 \\ \hline
\end{tabular}%}
\end{center}
\label{tab:publicacao}
\end{table}

\section{Como escrever equações}\label{s:Equações}

Veja a seguir exemplos equações criadas no latex.

\begin{equation}\label{eq:ACO1}
p_{ij}^{k}=\left\{\begin{matrix}
\frac{\tau _{ij}^{\alpha}(t)}{\sum_{j\in N_{i}^{k}}\tau_{ij}^{\alpha}(t)} \; \text{ se } j\in N_{i}^{k} \\
0 \; \text{ se } j\notin N_{i}^{k}
\end{matrix}\right.
\end{equation}

\begin{equation}\label{eq:ACO2}
\Delta \tau_{ij}^{k}(t) \propto ~\frac{1}{NSaltos^{k}(t)}.
\end{equation}

\begin{equation}\label{eq:ACO3}
\tau _{ij}(t+1) = \tau _{ij}(t)+\sum_{k=1}^{n_{k}}\Delta \tau _{i,j}^{k}(t).
\end{equation}

\begin{equation}\label{eq:ACO4}
\tau _{ij}(t+\Delta{t}) = (1-\rho)\tau_{ij}(t).
\end{equation}

\section{Como escrever algoritmos}\label{s:Algoritmos}

O Algoritmo \ref{alg:algS-ACO} apresenta o exemplo de um pseudo-código criado no latex. Para mudar o idioma dos comandos do algoritmo ou encontrar outros comandos, acesse o arquivo algorithmic.sty.

\begin{algorithm}
\caption{- Pseudocódigo do S-ACO \\ Fonte: Adaptado de \citeonline{ENGELBRECHT2007}}
\label{alg:algS-ACO}
\begin{algorithmic}[1]
\STATE Inicialize o tempo $t \leftarrow 0$
\STATE Inicialize o valor do feromônio $\tau_{i,j}(0)$
\STATE Crie $K$ formigas no nó de origem
\REPEAT
\FOR{cada formiga $k=1$ to $K$}
\STATE \COMMENT{Construindo um caminho $x^{k}(t)$}
\STATE $x^{k}(t) \leftarrow \emptyset$
\REPEAT
\STATE Selecione o próximo salto de acordo com a probabilidade da Equação~\ref{eq:ACO1}
\STATE Adicione o enlace $(i,j)$ ao caminho $x^{k}(t)$
\UNTIL {o destino seja alcançado}
\STATE Remova todos os \emph{loops} do caminho $x^{k}(t)$
\STATE Calcule a qualidade do caminho $f(x^{k}(t))$
\ENDFOR
\FORALL{enlace $(i,j)$ do grafo}
\STATE \COMMENT{Evaporação do Feromônio}
\STATE Reduza o feromônio $\tau_{i,j}(t)$ usando a Equação~\ref{eq:ACO4}
\ENDFOR
\FOR{formiga $k=1$ até $K$}
\FORALL{enlace $(i,j)$ de $x^{k}(t)$}
\STATE Atualize a quantidade de feromônio $\tau_{i,j}(t)$ usando as Equações~\ref{eq:ACO2} e~\ref{eq:ACO3}
\ENDFOR
\ENDFOR
\STATE $t \leftarrow t+1$
\UNTIL{que a condição de parada seja satisfeita}
\RETURN <caminho $x^{k}(t)$ de melhor qualidade como a solução>
\end{algorithmic}
\end{algorithm}