\chapter{Metodologia}\label{cap:metodologia}

Neste capítulo apresenta-se a metodologia a ser empregada a fim de alcançar os desígnios do projeto. Para tanto, apresenta-se de que forma o desenvolvimento será abordado pela equipe e, na sequência, as tecnologias a serem utilizadas, tais como o ambiente de programação, a linguagem de programação, mecanismos para controle de versão e, por fim, a plataforma de \textit{software}. 

\section{Abordagem da equipe}\label{s:fundamentos}

Partindo de ideias de aplicativos já existentes para acompanhamento de tráfego (Waze) e acompanhamento de ônibus na cidade de Curitiba (Busão Curitibano), a equipe decidiu reuni-las e aperfeiçoá-las em um aplicativo único. O aperfeiçoamento será resultado do uso de arquiteturas de rede que dispensam centralização, como as Redes Tolerantes a Atrasos, (DTN) se contrapondo à arquitetura cliente-servidor, na qual operam softwares (aplicativos) móveis existentes já citados no início desta seção. 

Sendo assim, o uso de uma arquitetura DTN dispensa a necessidade de conectividade contínua para transmissão de informações, uma vez que essa transmissão ocorrerá assim que um usuário, que necessita de informações, esteja próximo de um usuário que porte essas informações. 

Tendo em vista a grande quantidade de usuários que estarão utilizando o aplicativo, e o emprego de uma arquitetura descentralizada como as DTNs, nota-se também a importância da arquitetura Ponto-a-ponto (P2P), que permite que um usuário (que pode ser representado aqui por um nó da rede) adquira um papel simultâneo de "cliente" e "servidor". Em outras palavras, um usuário receberá/enviará informações de/para outros usuários, partindo do princípio que todos estejam utilizando o mesmo aplicativo móvel, a ser desenvolvido neste projeto. A descoberta de outros nós nas proximidades, bem como a propagação de informações por esses nós, dar-se-á de forma transparente ao usuário.

%\subsection{Arquitetura DTN e arquitetura P2P}

%Sistemas \sigla{P2P}{\emph{Peer-to-Peer}} (\emph{Peer-to-Peer}) são capazes de fornecer recursos de rede com sobreposição e auto-organização distribuída, com o intuito de prover uma distribuição eficiente dos dados. O mecanismo básico de funcionamento consiste em pares, nos quais uma entidade presta e ao mesmo tempo consume os recursos oferecidos por outras entidades que compõe o sistema. Outras características que estes sistemas apresentam, normalmente, são auto-organização, tolerância a falhas e escalabilidade \cite{italianos}.

%Em contrapartida, sistemas baseados em \sigla{DTN}{\emph{Delay Tolerant Networks}} (\emph{Delay Tolerant Networks}) são construídos para operar sobre redes sem fio, com múltiplos saltos. A arquitetura visa oferecer conectividade mesmo se uma comunicação fim a fim não exista por um (in)determinado período de tempo e, assim sendo, os nós intermediários da rede devem armazenar os dados até uma nova oportunidade de comunicação seja possível \cite{italianos}.

%Um paradigma típico para as DTNs é o chamado \emph{store-carry-forward} (armazenar, transportar e encaminhar) que consiste na armazenagem de uma mensagem em um nó, o transporte da mesma por este nó e a sua posterior distribuição para outros nós da rede, tão logo surja uma oportunidade para tal \cite{italianos}. Um exemplo simples é exposto na sequência. Supondo que um nó \textbf{S} tenha que se comunicar com um nó \textbf{D}, mas não existe um caminho direto entre os dois; por consequência disso, o nó \textbf{S} terá que armazenar a mensagem. No entanto, \textbf{S} consegue se comunicar com um terceiro nó \textbf{R}, que por sua vez se comunica com o nó \textbf{D}. O nó \textbf{S} pode então encaminhar a mensagem para o nó \textbf{R}, que irá repassá-la para o nó \textbf{D} \cite{italianos}.

%Como consequência dos dois mecanismos apresentados, no escopo deste exposto decidiu-se pelo uso de uma arquitetura de rede que combina os dois modelos apresentados anteriormente. Apesar de algumas disparidades entre os dois paradigmas (a mais notável: operarem em camadas distintas de rede) \cite{italianos}, usar-se-á os dois mecanismos em conjunto a fim de oferecer uma arquitetura de rede decentralizada e com boa conectividade a todos os supostos usuários do aplicativo. Ademais, o fato de ambos trabalharem em ambientes decentralizados, auto-organizados e distribuídos \cite{italianos} tornam-se bastantes interessantes para o escopo do trabalho, já que se encaixa perfeitamente no ambiente em que o aplicativo desenvolvido funcionará. 

%Outra característica interessante, consiste na possibilidade de operação em mecanismos de rede que não são baseados em meios físicos para propagar informações, como Wi-Fi e Bluetooth, por exemplo. Essa possibilidade é ideal para smartphones, dado que uma gama abundante desses aparelhos possui no mínimo um desses recursos disponíveis. 

\section{Tecnologias}\label{s:tecnologias}

A presente seção refere-se às principais tecnologias a serem empregadas no desenvolvimento do projeto, tais como ambientes de desenvolvimento, linguagens de programação, plataforma de \textit{software} e plataforma de \textit{hardware}.

\subsection{Ambiente de desenvolvimento}

Pelo fato de o projeto envolver um aplicativo para o sistema operacional Android, é necessário utilizar um ambiente de desenvolvimento específico para essa plataforma.

O \emph{site} oficial para desenvolvimento de aplicativos Android, Android Developers\footnote{O site pode ser acessado em http://developer.android.com/}, disponibiliza algumas ferramentas para a concepção e teste de aplicativos para a plataforma.

É possível utilizar um \emph{plug-in}, juntamente com o Android SDK, e integrá-los em uma instalação existente da conhecida IDE chamada Eclipse, realizando as devidas configurações manualmente. Uma segunda opção, e que torna a preparação do ambiente de desenvolvimento mais simplificada, é baixar um \emph{bundle} que já inclui o Android SDK, a IDE Eclipse, o \emph{plug-in}, ferramentas de desenvolvimento e uma imagem de sistema para emular um dispositivo Android no \textit{Desktop}. Dessa forma, é possível depurar aplicações Android diretamente no PC, antes de enviá-lo à um dispositivo real. A terceira opção é utilizar uma IDE alternativa chamada \emph{Android Studio}, que atualmente encontra-se em uma versão Beta.

Por experiências dos integrantes com o referido \emph{bundle}, em projetos anteriores, optou-se por utilizar o mesmo no desenvolvimento deste projeto.

Para testes de software será utilizada a emulação do dispositivo Android provida pelo \emph{bundle}, até onde for possível. É dito isso pois o projeto envolve o uso de GPS, Google Maps e compartilhamento de informações entre usuários (\emph{crowdsourcing}), o que não é possível de ser feito apenas com a utilização de um emulador. Dessa forma, será necessário testar a aplicação em um dispositivo real, como um \emph{smartphone} ou um \emph{tablet}, com sistema operacional Android.

\subsection{Linguagem de programação}

Em relação à linguagem de programação, não há a necessidade de fazer um levantamento dentre as linguagens de programação existentes atualmente e escolher qual será a mais adequada ao projeto. Uma vez que o projeto tem como foco o desenvolvimento de uma aplicação Android, é mandatório que a mesma seja desenvolvida utilizando a linguagem de programação Java juntamente com a API (que contém bibliotecas e ferramentas) incluída no Android SDK. A princípio, não será utilizada nenhuma biblioteca de terceiros (\emph{third-party}).

\subsection{Controle de Versão}

Será utilizado um sistema de controle de versão (\sigla{CVS}{\emph{Control Version System}}, \emph{Control Version System}) para o versionamento do código-fonte do aplicativo Android. Vários são os sistemas de versionamento de arquivos, dentre os quais os mais comuns são o SVN e Git. Para este último, existem repositórios de código-fonte gratuitos, como o GitHub (https://github.com) e o BitBucket (https://bitbucket.org).

O ideal seria manter um servidor próprio com o SVN ou o Git instalados. No entanto, isso poderia envolver custos adicionais, pois seria necessário deixar uma máquina dedicada para o controle de versão e acessível para todos os integrantes do projeto, além de realizar \emph{backups} periódicos.

O GitHub permite que qualquer pessoa crie uma quantidade ilimitada de repositórios gratuitamente e compartilhe-o com quantos colaboradores for necessário. No entanto, o código-fonte fica disponível publicamente, para qualquer um acessar. Para projetos acadêmicos (como TCCs, teses de mestrado e doutorado), isso pode não ser indicado. O GitHub também oferece repositórios privados, com acesso restrito, mas apenas a partir da assinatura de um plano.

Em contrapartida, o BitBucket oferece uma quantidade ilimitada de repositórios privados gratuitamente, permitindo até cinco colaboradores em um projeto. Como o projeto é constituído de dois integrantes e levantou-se a necessidade de versionar o código-fonte sem disponibilizá-lo publicamente, optou-se por utilizar o BitBucket como repositório de código-fonte e o Git como como sistema de controle de versão.

\subsection{Plataforma de \textit{software}}

Em relação ao sistema operacional Android, levanta-se outra questão. O Android, como sistema operacional, possui diversas versões desde o seu lançamento. Cada uma dessas versões acaba por incluir novas funcionalidades não apenas para o usuário final, mas também, para os desenvolvedores. É certo que existe a chamada retrocompatibilidade, que permite que a grande maioria das aplicações desenvolvidas para uma versão mais recente do S.O., execute sem maiores problemas em uma versão mais antiga. No entanto, para que isso ocorra, muitas vezes a aplicação fica limitada, pois o desenvolvedor é forçado à desabilitar certos recursos que não são compatíveis, de forma alguma, com versões mais antigas da plataforma.

No momento da escrita deste documento, a versão mais recente do Android é a versão 4.4 (KitKat). A maioria dos dispositivos disponíveis no mercado -- \emph{smartphones} e \emph{tablets} -- possuem uma versão do Android 4.0+ instalada. A mais recente versão do S.O., a versão 5.0 (Lollipop), está prestes a ser lançada, e apenas alguns poucos dispositivos já existentes no mercado receberão a atualização para o novo sistema.

A Tabela \ref{tab:marketShare} fornece informações referentes ao \emph{market share} das versões do Android, desde a 2.2 (Froyo) até a 4.4 (KitKat). Os dados foram coletados através da aplicação Google Play Store em um período de sete dias, finalizando no dia 03 de novembro de 2014. É possível verificar o percentual de quantos dispositivos executam uma determinada versão do sistema operacional. Essas informações são úteis para que o desenvolvedor determine qual versão da plataforma terá como ``alvo''.

\begin{table}[h!]
%Se o nome da tabela for muito grande, é possível colocar entre colchetes o título reduzido que será mostrado na lista de tabelas e entre aspas o título completo que será mostrado acima da tabela. O mesmo é válido para figuras.

\caption[\textit{Market Share} das versões do Android]{\textit{Market Share} das versões do Android. Fonte: \cite{androidDevDashboards}}
\begin{center}
%\scalebox{0.75}{
\begin{tabular}{|c|c|c|c|}
\hline
\textbf{Versão}     &\textbf{Codinome}      & \textbf{API}      &\textbf{Distribuição} \\ \hline \hline
2.2                 & Froyo                 & 8                 & 0,6$\%$	\\
2.3.3 - 2.3.7       & Gingerbread           & 10                & 9,7$\%$	\\
4.0.3 - 4.0.4       & Ice Cream Sandwich    & 15                & 8,5$\%$	\\
4.1.x               & Jelly Bean            & 16                & 22,8$\%$	\\
4.2.x               & Jelly Bean            & 17                & 20,8$\%$	\\
4.3                 & Jelly Bean            & 18                & 7,3$\%$	\\
4.4                 & KitKat                & 19                & 30,2$\%$	\\ \hline
\end{tabular}%}
\end{center}
\label{tab:marketShare}
\end{table}

Observa-se na tabela que cerca de 10,4$\%$ usuários usa um dispositivo com uma versão antiga da plataforma, de codinome Froyo (2.2) ou Gingerbread (2.3.3 - 2.3.7). Uma parcela ainda menor utiliza as versões Ice Cream Sandwich (4.0.3 - 4.0.4), 8,5$\%$. Grande parte dos usuários -- cerca de 50$\%$ -- utiliza o Android Jelly Bean (4.1.x - 4.3) e 30,2$\%$ utiliza a última versão lançada, KitKat (4.4).

Portanto, devido à popularidade da versão Jelly Bean (4.1.x - 4.3) do sistema operacional Android, o desenvolvimento do projeto terá como foco dispositivos que executam, no mínimo, esta versão da plataforma.

\subsection{Plataforma de \textit{hardware}}

Para o projeto em questão não será necessária a utilização de uma plataforma de \textit{hardware}, pois a aplicação executará diretamente sobre um dispositivo móvel, tal como um \textit{smartphone} ou um \textit{tablet}.

\section{Etapas de desenvolvimento}

Inicialmente será feito um estudo das APIs disponíveis no Android SDK, e um levantamento para detectar se há a necessidade de utilizar APIs de terceiros (\textit{third-party}). 

Com os diagramas UML já projetados, será preparado o ambiente de desenvolvimento, realizando-se o \textit{download} do \textit{bundle}, instalação de \textit{drivers} para depuração em um dispositivo móvel real via cabo USB e configurações necessárias. Na sequência, dar-se-á início à implementação da aplicação.

Inicialmente, os testes serão realizados no emulador Android, disponibilizado pelo \textit{bundle}, conforme citado na seção anterior. Testes mais complexos serão feitos em um dispositivo real (\textit{smartphone} e/ou \textit{tablet}) que execute o sistema operacional Android com versão 4.1.x, pelo menos.