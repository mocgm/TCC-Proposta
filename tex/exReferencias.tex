%---------- Referencias ----------

\chapter{Referências Bibliográficas}

Todas as referências citadas no texto devem estar relacionadas em ordem alfabética. \textbf{Ao utilizar o Latex, as referências são ordenadas automaticamente.}

Referências não citadas no texto do documento, não devem ser apresentadas na lista de Referências Bibliográficas.

As normas da ABNT para referências bibliográficas podem ser consultadas em \cite{NormasUTFPR}.

Para citar referências no texto utilizando o latex, use o comando $citeonline$ ou $cite$, precedido de uma barra invertida.

Com o comando $citeonline\{vendramin2012\}$, a impressão no texto ficará da seguinte forma:

Vendramin (2012).

Ou seja, a citação aparece inserida dentro do texto, como por exemplo:

Conforme a autora Vendramin (2012), a proposta de TCC ...

Ao usar o comando $cite\{vendramin2012\}$, a impressão no texto ficará da seguinte forma:

A proposta de TCC ... (Vendramin, 2012).

A seguir são apresentados exemplos de referências bibliográficas criadas no Latex. As referências são inseridas em um arquivo .bib e o nome desse arquivo (por exemplo, Referencias) deve ser usado ao chamar o comando $bibliography\{Referencias\}$ precedido por uma barra invertida. O comando $bibliography$ é responsável por gerar de forma automática, a partir do arquivo $Referencias.bib$, as referências bibliográficas citadas no documento.

\textbf{Livros}

Para exemplos de como referenciar livros veja as seguintes referências: \cite{castro2006} e \cite{ENGELBRECHT2007}.

\textbf{Artigos de Periódicos}

Para exemplos de como referenciar artigos de periódicos veja as seguintes referências: \cite{vendramin2012a} e \cite{cao2012}.

\textbf{Relatórios Técnicos e Normas Técnicas}

Para exemplos de como referenciar relatórios técnicos ou normas técnicas veja as seguintes referências: \cite{mills2004} e \cite{RFC5050}.

\textbf{Artigos Científicos em Congressos/Conferências/Simpósios}

Para exemplos de como referenciar artigos científicos veja as seguintes referências: \cite{vendramin2012b} e \cite{vendramin2011}.

\textbf{Monografias/Dissertações/Teses}

Para exemplos de como referenciar monografias, dissertações ou teses veja as seguintes referências: \cite{vendramin2012c} e \cite{saleem2001}.