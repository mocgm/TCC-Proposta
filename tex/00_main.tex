
% Proposta TCC1 - Professora Ana Cristina B. Kochem Vendramin (cristina@dainf.ct.utfpr.edu.br, criskochem@utfpr.edu.br)

\documentclass{normas-utf-tex_07_2012} %Estilo de Formato criado pelo Prof. Hugo Vieira Neto (hvieir@utfpr.edu.br)
%Se você ainda não conhece o latex, comece olhando o site do Prof. Hugo --> http://pessoal.utfpr.edu.br/hvieir/orient/

%\documentclass[twoside,openright]{normas-utf-tex} %openright = o capitulo comeca sempre em paginas impares

%\usepackage[alf,abnt-emphasize=bf,bibjustif,recuo=0cm, abnt-etal-cite=2, abnt-etal-list=99]{abntcite} %configuracao das referencias bibliograficas.
\usepackage[alf]{abntex2cite}	% Citações padrão ABNT
\usepackage[brazil]{babel} % pacote portugues brasileiro
\usepackage[utf8]{inputenc} % pacote para acentuacao direta
\usepackage{amsmath,amsfonts,amssymb} % pacote matematico
\usepackage[pdftex]{graphicx} % pacote grafico
\usepackage{times} % fonte times
\usepackage{a4wide}
\usepackage[a4paper]{geometry} %define papel a4...
\geometry{left=3cm,right=2cm,top=3cm,bottom=2cm} % ...e suas margens
\usepackage{tabularx,multirow,longtable} %pacotes para mesclar linhas/colunas; tabelas grandes
\usepackage{fancyhdr} % altera cabeçalhos
\usepackage[T1]{fontenc} % acentuação direto no texto
\usepackage{ae} % “Almost European”: aumenta a qualidade do pdf final
\usepackage{rotating} % faz rotações de tabelas e figuras
\usepackage{indentfirst} % tabula a primeira linha do parágrafo
\usepackage[hang,small,bf]{caption} % legendas; nome "tabela" ou "figura" em negrito
\usepackage{caption3}
\usepackage{setspace} % ajuste do espaço entre-linhas
%\usepackage{algorithm}
\usepackage{subfig}
\usepackage{float} % posicionamento das figuras
\usepackage{subfloat}
%\usepackage{arydshln}
%\usepackage[backend=biber]{biblatex}
\makeatletter
%\renewcommand{\ALG@name}{Algoritmo}
%\usepackage{algorithmic}
\usepackage{array,ragged2e}

%para a hifenacao funcionar é necessario fazer a seguinte modificacao
%Clique em Iniciar -> Programas -> MikTeX -> MiKTeX options Vá em Languages e marque a caixa português.

%hifenacoes customizadas
\hyphenation{mo-de-lar re-no-vá-veis re-pre-sen-tar re-pre-sen-ta-ção la-te-rais res-pec-ti-vos re-a-ção re-a-li-zan-do o-pe-ra-ções cons-tan-te di-fe-ren-tes tem-pe-ra-tu-ra ex-tre-mi-da-de ter-mo-di-nâ-mi-ca trans-es-te-ri-fi-ca-ção In-ter-net}

%%%% Comandos introduzidos para controlar as alteracoes no fonte%%%%%%%%%%%%%%%%%
% assigning colors to comments of each author
\usepackage{color}
\newcommand{\cris}[1]{\textcolor{blue}{#1}} %texto final
\newcommand{\crisC}[1]{[\textcolor{blue}{#1}]} % comentário entre colchetes
\newcommand{\ane}[1]{\textcolor{red}{#1}} %texto final
\newcommand{\aneC}[1]{[\textcolor{red}{#1}]} % comentário entre colchetes
%%%%%%%%%%%%%%%%%%%%%%%%%%%%%%%%%%%%%%%%%%%%%%%%%%%%%%%%%%%%%%%%%%%%%%%%%%%%%%%%%%%%%%

% ---------- Preambulo ----------
\instituicao{Universidade Tecnológica Federal do Paraná} % nome da instituicao
\programa{Departamento Acadêmico de Informática} % nome do programa ou departamento
\area{Curso de Engenharia de Computação} % área ou curso

\documento{Trabalho de Conclusão de Curso} % [Trabalho de Conclus\~ao de Curso] ou [Disserta\c{c}\~ao] ou [Tese]
\nivel{Graduação} % [Gradua\c{c}\~ao], [Especializa\c{c}\~ao], [Mestrado] ou [Doutorado]
\titulacao{Engenheiro} % [Engenheiro], [Tecn\'ologo], [Bacharel], [Mestre] ou [Doutor]

\titulo{\MakeUppercase{Utilização de aplicativo móvel e \textit{crowdsourcing} para rastreio de ônibus}} % titulo do trabalho em portugues
\title{\MakeUppercase{}} % titulo do trabalho em ingles

\autor{Gionatta Marcon Mocellin} % autor do trabalho
\autordois{Giusepe Pietro Niquele}
\cita{MOCELLIN, Gionatta Marcon. NIQUELE, Giusepe Pietro} % sobrenome (maiusculas), nome do autor do trabalho

\palavraschave{Dispositivos móveis, Redes tolerantes a atraso, Cidades inteligentes, \textit{Crowdsourcing}, Acompanhamento de ônibus} %substituir pelas palavras-chave relacionadas ao seu tema de pesquisa
\keywords{Mobile devices, Delay Tolerant Networks, Smart Cities, Crowdsourcing, Bus tracking} % incluir palavras-chave em inglês

\comentario{\UTFPRdocumentodata\ de Engenharia de Computação apresentado ao \UTFPRprogramadata\ da \ABNTinstituicaodata\ como requisito parcial para obtenção do título de ``\UTFPRtitulacaodata\ em Computação''.} %\\ \'Area de Concentra\c{c}\~ao: \UTFPRareadata.}

\orientador[Orientadora:]{Prof$^a$. Dr$^a$. Ana Cristina Kochem Vendramin} % <- no caso de orientadora, usar esta sintaxe
\coorientador[Co-orientadora:]{\hspace*{3mm}Prof$^a$. Dr$^a$. Anelise Munaretto \newline \hspace*{3mm}Fonseca} % <- no caso de co-orientadora, usar esta sintaxe


\local{Curitiba} % cidade
\data{2014} % ano
\setcounter{tocdepth}{4} % para numeração das seções
\setcounter{secnumdepth}{4}

\numberwithin{equation}{chapter} %
%\numberwithin{table}{chapter} %
%\numberwithin{figure}{chapter} %

\makeatletter  %% this is crucial
 \renewcommand\subsubsection{\@startsection{subsubsection}{3}{\z@}%
                        {-2\p@ \@plus -0.5\p@ \@minus -0.5\p@}%
                        %{8\p@ \@plus 4\p@ \@minus 4\p@}%     <-- this is copied from the subsection command
                        {2\p@ \@plus 1\p@ \@minus 1\p@}%     <-- this is copied from the subsection command
                        {\normalfont\normalsize\bfseries\boldmath
                         \rightskip=\z@ \@plus 8em
 \pretolerance=10000 }}
\makeatother   %% this is crucial

%---------- Inicio do Documento ----------
\begin{document}

\capa % geracao automatica da capa

\folhaderosto % geracao automatica da folha de rosto

% dedicatória (opcional)
%\begin{dedicatoria}
% Texto da dedicat\'oria.
%\end{dedicatoria}

% agradecimentos (opcional)
%\begin{agradecimentos}
% Texto dos agradecimentos.
%\end{agradecimentos}

% epigrafe (opcional)
%\begin{epigrafe}
%\end{epigrafe}

%resumo
\begin{resumo}

O incessante ritmo ditado pelos grandes centros urbanos gera na população a necessidade de minimizar o tempo gasto em seus deslocamentos, principalmente quando o transporte público é utilizado para este propósito. Contudo, é muito difícil determinar com precisão quando um ônibus chegará ao ponto ou ao terminal. Por vezes, isso causa longos períodos de espera, ou porque o ônibus acabou de partir, ou porque se atrasou devido a um imprevisto. Felizmente, a tecnologia evolui a tal ponto que os chamados \textit{smartphones} se tornaram um artefato muito popular. Ainda, o advento de redes móveis e sem fio, tais como o 3G e o WiFi, permite que pessoas fiquem conectadas a maior parte do tempo, independente do lugar em que estejam. Dentre os inúmeros benefícios que os \textit{smartphones} fornecem, a possibilidade de troca de informações entre usuários é uma delas, permitindo um fenômeno chamado \textit{crowdsourcing}. O \textit{crowdsourcing}, aplicado em um âmbito maior - que englobe toda ou a maior parte da cidade - poderia ser um primeiro passo para elevá-la ao nível de uma \textit{Smart City} ou ``cidade inteligente'', onde a tecnologia é utilizada pela população para a solução/mitigação de um determinado problema urbano, no caso, a mobilidade. Neste projeto é proposto o desenvolvimento de um aplicativo móvel que alie conceitos, tanto de \textit{Smart Cities} quanto de \textit{crowdsourcing}, no qual informações em tempo real sobre o transporte coletivo, especificamente no que compete o estado atual de uma linha de ônibus, sejam disponibilizadas aos usuários.

\end{resumo}

%abstract
\begin{abstract}

The unceasing rhythm dictated by the large urban centers causes in the population the necessity of minimizing the time spent in their locomotion, especially when the public transport is used for this purpose. However, it's difficult to determine with precision when the bus will stop at the bus station. Fortunately, the technology evolved to a point where the smartphones are now a very popular artifact. Nevertheless, the mobile and wireless networks, such as 3G and WiFi, allow the people to stay connected all the time, regardless the place they are. Amongst the severeal benefits brought by the smartphones, the information exchange between the users is one of them, making possible a phenomenon called crowdsourcing. The crowdsourcing, applied in a larger environment -- covering the entire, or almost the entire, city -- may be the first step to evolve it to a Smart City, where the technology is used by the population to solve/mitigate a specific problem, such as mobility. In this project is proposed the development of a mobile application which integrates concepts from Smart Cities and crowdsourcing, where real time informations concerning the public transport, specifically the actual state of a bus line, is availabe to the users. 
%Include the abstract here

\end{abstract}

% listas que recomenda-se a partir de 5 elementos
%\listadefiguras % geracao automatica da lista de figuras

%\listadetabelas % geracao automatica da lista de tabelas

\listadesiglas % geracao automatica da lista de siglas
%Constituída de uma relação alfabética das abreviaturas e siglas utilizadas no texto, seguido das palavras ou expressões correspondentes grafadas por extenso. Utilizada apenas se houver siglas.

%\listadesimbolos % geracao automatica da lista de simbolos
%Elaborado de acordo com a ordem apresentada no texto, seguido do significado correspondente. Utilizada apenas se houver símbolos.

% sumario
\sumario % geracao automatica do sumario

%---------- Inicio do Texto ----------

\setcounter{page}{3} % *** Necessário arrumar manualmente antes de imprimir

% O label serve para referenciar o capítulo quando necessário
\chapter{Introdução}\label{cap:introducao}

O conceito de \textit{Smart Cities} (Cidades Inteligentes, em tradução livre) é um sustentáculo elementar para o desenvolvimento urbano sustentável. De certa forma, ele poderia contribuir na solução de diversos problemas críticos, originados a partir da urbanização de grandes cidades como congestionamentos, poluição do meio ambiente, e os limites dos recursos naturais. \cite{pan2013}.
 
Um outro aspecto do conceito de cidades inteligentes a ser considerado, é a importância crescente das tecnologias digitais para um futuro sustentável. Assim sendo, destaca-se seis principais áreas, nas quais essas inovações digitais tem a competência para fazer a diferença: vida inteligente, governança inteligente, economia inteligente, ambiente inteligente, pessoas inteligentes e mobilidade inteligente \cite{schuurman}.
 
Com o uso da tecnologia da informação e comunicação (do inglês {\sigla{ICT}{\emph{Information and Communication Technology}}, \emph{Information and Communication Technology},}) a  análise e mineração de dados de sensoriamento grande cidades é um passo importante para considerá-las uma cidade inteligente. Por exemplo, informações sobre a mobilidade dos veículos e pessoas tornaram-se temas em voga nos últimos tempos  devido à prevalência do \sigla{GPS}{\emph{Global Positioning System}} (\emph{Global Positioning System}) e outras tecnologias de localização. Esse conhecimento sobre a cidade e seus habitantes poderá trazer benefícios nas áreas de transporte, planejamento urbano, saúde pública, segurança pública e comércio \cite{pan2013}.
 
Embora as ações de sensoriamento e mineração dos dados seja de imprescindível importância, algo deve ser feito para transformar esses dados em informações. Nesse sentido, uma abordagem interessante para a resolução desta questão consiste no conceito de \textit{crowdsourcing} \cite{schuurman}.

\citeonline{Howe2006} caracteriza \textit{crowdsourcing} como o fenômeno no qual várias pessoas comuns dedicam seu tempo livre na busca de soluções para um problema de interesse coletivo. Isto posto, a massificação dos chamados \textit{smartphones} e o advento de redes móveis e sem fio tais como o 3G e o WiFi e recentemente, o 4G, permitem que pessoas fiquem conectadas a maior parte do tempo, em qualquer lugar. Dentre os inúmeros benefícios que os \textit{smartphones} fornecem, a possibilidade de troca de informações entre usuários é uma delas, bem como a aplicação do conceito de \textit{crowdsourcing}.

Considera-se o seguinte cenário: uma pessoa que acabou de chegar em um ponto de ônibus, deseja saber onde o mesmo se encontra. Questões como ``será que o ônibus está chegando?'', ``será que vai demorar?'' ou ``será que o ônibus estragou?'' são bastante comuns. Mas como descobrir suas respostas? Infelizmente, não é possível respondê-las apenas com a tabela de horários de chegada dos ônibus, disponível nos terminais e no \textit{site} da \sigla{URBS}{Companhia de Urbanização e Saneamento de Curitiba} (Companhia de Urbanização e Saneamento de Curitiba).

Nesse sentido, é proposto nesse documento uma solução possível para esse quesito. A ideia básica é a construção de um aplicativo que alie conceitos, tanto de \textit{Smart Cities} quanto de \textit{crowdsourcing}, no qual informações em tempo real sobre o transporte coletivo, especificamente no que compete o estado atual de uma linha de ônibus, sejam disponibilizadas aos usuários.

%breve estado da arte

Alguns aplicativos para dispositivos móveis foram desenvolvidos com o objetivo de prover, ao usuário, informações referentes ao transporte público. Um dos aplicativos mais recentes é o chamado ``Busão Curitibano'', lançado em 2013\footnote{Para mais informações sobre o aplicativo, consultar notícia da Gazeta do Povo, disponível em:  http://www.gazetadopovo.com.br/vidaecidadania/conteudo.phtml?id=1377575}. Os desenvolvedores -- e também os usuários do aplicativo -- encontraram algumas resistências por parte da URBS, pois o aplicativo usava a base de dados do transporte coletivo. Devido aos inúmeros acessos, acabava derrubando o servidor da URBS e tornando o serviço de acompanhamento de ônibus indisponível.

Outro aplicativo semelhante foi desenvolvido por \citeonline{sujatha}. Os autores propuseram um sistema chamado \sigla{MBTS}{Mobile Bus Tracking System} (\emph{Mobile Bus Tracking System}) que ajuda qualquer pessoa a obter informações de um determinado ônibus sem fazer ligações ou perturbar passageiros com mensagens \sigla{SMS}{\emph{Short Message Service}} (\emph{Short Message Service}). %O sistema funciona com, basicamente, três atores: os passageiros, os “coordenadores” do ônibus e as pessoas que estão aguardando em um ponto. Todos os atores necessitam de um \textit{smartphone} com sistema operacional Android conectado à Internet. Continuamente, os passageiros e os ``coordenadores'' alimentam um banco de dados com informações sobre o ônibus, tais como as coordenadas do veículo, as quais estarão disponíveis para as pessoas que estão aguardando pelo ônibus.
Um banco de dados é utilizado para armazenar informações sobre os ônibus (por exemplo, as coordenadas do veículo), as quais estarão disponíveis para acesso através de um aplicativo Android..

\section{Justificativa}

A utilização de uma arquitetura centralizada, estilo cliente--servidor é um ponto comum nos dois trabalhos relacionados. Em ambos, nota-se que a arquitetura encontra-se em uma relação n:1, ou seja, vários clientes para um único servidor. Alguns dos problemas que podem surgir nesse tipo de arquitetura é que o servidor pode ficar sobrecarregado e eventualmente ficar indisponível devido ao grande número de usuários utilizando o serviço, como foi o caso do ``Busão Curitibano''. Além disso, nesse caso, o servidor utilizado não é de propriedade dos desenvolvedores, mas sim da Urbs. Caso a companhia deseje interromper o serviço de fornecimento de dados, o ``Busão Curitibano'' perde sua utilidade. Isso não constitui um problema para o MBTS, pois o mesmo concentra as informações em um servidor próprio, mantido pelos próprios desenvolvedores.

Sendo assim, a principal motivação para o desenvolvimento deste projeto é implementar um aplicativo de acompanhamento de ônibus para dispositivos móveis -- \textit{smartphones} -- que utilize uma rede descentralizada. Nela, os próprios usuários são os nós, que compartilharão informações entre si. Para tal, utilizar como referência o aplicativo móvel ``Waze''\footnote{Uma análise do aplicativo pode ser encontrada em http://www.techtudo.com.br/tudo-sobre/waze.html}, que implementa uma espécie de comunidade para mapeamento do trânsito em tempo real. Através do ``Waze'', cerca de 40 milhões de usuários compartilham informações, no mundo todo \cite{techtudoWaze}.

Além disso, usufruir do recurso de GPS, já utilizado nos ônibus de Curitiba, para fornecer aos usuários a posição dos veículos.

\section{Objetivo geral}

Desenvolver um aplicativo móvel que execute sobre o sistema operacional Android, aliando conceitos de \textit{Smart Cities}
e \textit{Crowdsourcing}, no qual informações em tempo real sobre o transporte coletivo sejam disponibilizadas aos usuários.

%Desenvolver um aplicativo para \textit{smartphones} que rodam sobre o sistema operacional Android, para acompanhamento de ônibus em Curitiba, unindo idéias de aplicativos já existentes como o ``Waze'' e o ``Busão Curitibano'', e o conceito de \emph{crowdsourcing}.

\section{Objetivos específicos}

\begin{itemize}
%\item Desenvolver um aplicativo, para \textit{smartphones}, para acompanhamento de ônibus da rede de transporte de Curitiba;
\item Desenvolver um aplicativo móvel, exclusivamente para \textit{smartphones} e \textit{tablets} com sistema operacional Android, para acompanhamento de ônibus da rede de transporte de Curitiba;
\item Estudar e implementar no aplicativo o conceito de \textit{crowdsourcing}, para permitir que usuários colaborem entre si com informações pertinentes, através de \textit{feedbacks} pré-definidos no aplicativo;
\item Desenvolver o aplicativo para que o mesmo seja executado sobre uma plataforma que não exija custos de desenvolvimento;
\item Implementar uma arquitetura de rede descentralizada, a fim de evitar problemas como sobrecarga de servidor.
\item Unir ideias de aplicativos móveis já existentes, como o ``Waze'' e o ``Busão Curitibano'', no desenvolvimento do projeto.
\end{itemize}

\section{Estrutura e organização}

O presente documento encontra-se organizado da seguinte forma. O capítulo 1 apresenta a introdução, com a formulação do problema, justificativa do projeto, objetivo geral e específicos do mesmo. O capítulo 2 apresenta o levantamento bibliográfico, importante para fundamentar o projeto e fornecer um embasamento. A metodologia, que descreve as etapas de desenvolvimento pelas quais o projeto percorrerá, é apresentada no capítulo 3. Os recursos para desenvolvimento do projeto e a viabilidade do mesmo, são apresentados nos capítulos 4 e 5, respectivamente. Por fim, conclusões pertinentes ao desenvolvimento do projeto, tais como dificuldades esperadas e horizontes de projeto, são apresentados no capítulo 6.

\chapter{Levantamento Bibliográfico}\label{cap:levbibliog}

Este capítulo apresenta uma breve introdução à Smart Cities, Crowdsourcing, contexto atual do desenvolvimento de programação voltada a aplicativos de transporte público e demais conceitos, pormenores, ao desenvolvimento do trabalho.

\section{Smart Cities}

O conceito de Cidades Inteligentes e o respectivo aumento no seu uso, pode ser reconhecido como o resultado do alto crescimento das tecnologias digitais na intenção de garantir um futuro sustentável. Embora o conceito seja empregado normalmente com o objetivo de elaborar estratégias que objetivem melhorar a qualidade de vida dos cidadãos, criando um futuro sustentável, o conceito por si só continua sendo usado em diferentes contextos e assim permanece um tanto ambíguo. Dessa forma faz-se necessário estabelecer um critério muito importante que o diferencia das cidades-conceito, o aspecto colaborativo entre os diversos interessados, mais comumente os cidadãos \cite{schuurman}. 

Um outro enfoque, mais empresarial, apresentado por \citeonline{walravens}, menciona que o conceito também pode ser empregado para caracterizar um grupo de organizações que intentam por revolucionar algum aspecto em uma região, entre os quais estão parques empresariais, o nível de escolaridade de uma população, o uso de tecnologias em contextos urbanos, o aumento da eficácia de governos e especialmente aquelas com foco em ITC. Walravens argumenta também que de maneira geral, o uso de serviços de telefonia móvel adquirem uma importância inevitável, já que é um sub aspecto de ITC e dessa forma grande importância na construção de uma Cidade Inteligente \cite{walravens}.

Em vista disso, sistemas operacionais desenvolvidos para \emph{smartphones} inspiram desenvolvedores a conceber aplicativos e serviços que aperfeiçoam a vida nas cidades de diversas formas diferentes, como por exemplo, facilitando acesso à informação sobre o transporte publico. Além disto, à medida que os \emph{smartphones} se tornam mais acessíveis e populares, cria-se o desejo de que se tornem eminentes ferramentas na busca de uma cidade mais inteligente \cite{walravens}.




%A fundamentação teórica atribui, essencialmente, credibilidade ao trabalho, faz referência às pesquisas e aos conhecimentos já construídos e publicados, situando a evolução do assunto e, assim, dando sustentação ao tema que está sendo estudado.

%É preciso situar historicamente a evolução do tema, quais as abordagens já investigadas, qual o estágio atual do conhecimento sobre o assunto ou quais as tendências que se apresentam.

%É necessário apresentar uma análise do estado da arte do problema abordado. Não se trata de uma simples transcrição de pequenos textos ou citações, mas sim de uma sistematização de ideias, fundamentos, conceitos e proposições de vários autores, apresentados de forma lógica, encadeada e descritiva, demonstrando que foram estudados e analisados pelo autor.

%Deve-se realizar o levantamento bibliográfico junto a diferentes fontes documentais, como livros, obras de referência, periódicos científicos, teses, dissertações, monografias, artigos, dentre outros.

%inicio

%Um aplicativo que aproxima-se bastante do projeto proposto é o chamado ``Waze''. O aplicativo foi comprado pela Google em 2013 por cerca de US$\$$ 1,3 bilhões. Cohan aponta quatro grandes motivos para a compra do aplicativo pela Google.

%Segundo Cohan, um dos motivos é o compromisso que um usuário adquire ao utilizar o Waze. A comunidade de usuários beira os 50 milhões; são 50 milhões de usuários contribuindo com informações sobre o trânsito, em praticamente qualquer local do planeta, à todo momento. A comunidade atribui espécies de ``recompensas'', através de pontuações e insígnias, à usuários que realmente são comprometidos em ajudar os outros (COHAN). A Google pode aproveitar esse esquema de comunidades de usuários e \emph{crowdsourcing} para aplicar em futuros projetos. 

%Outro fator importante é que o Waze acaba sendo um complemento ao serviço Google Maps, bastante popular. O aplicativo permite que usuários notifiquem os outros sobre acidentes, radares e ruas bloqueadas, por exemplo, através da utilização de \emph{crowdsourcing}. Para Cohan, o Waze apresenta-se não apenas como um complemento, mas também uma alternativa ao tradicional Google Maps. 

%(TODO: INCLUIR FONTE DO COHAN http://www.forbes.com/sites/petercohan/2013/06/11/four-reasons-for-google-to-buy-waze/)

%(TODO: DESCREVER FUNCIONAMENTO DO WAZE)

\section{Crowdsourcing}

Outro aspecto importante  a ser mencionado é a caracterização de \emph{crowdsourcing}, no que tange inteligência coletiva. À grosso modo, entende-se inteligência coletiva como quando indivíduos de um determinado grupo, que possui interesse por determinado assunto, combinam seus conhecimentos a fim de encontrar a solução para um determinado problema.  Através da interação social, o conhecimento individual é compartilhado, corrigido, processado, enriquecido e avaliado. Normalmente, os resultados colhidos são melhores que os obtidos por um único indivíduo. Talvez a aplicação mais difundida desse conceito seja a Wikipedia  \cite{schuurman}. 

Por conseguinte, a ponte que se estabelece entre \emph{crowdsourcing} e \emph{smart cities} acompanha o advento dos smartfones. Para \citeonline{kanhere}, as melhorias no poder de processamento, sensoriamento e capacidades de armazenamento permite que telefones celulares sejam comparados a dispositivos de computação. Esse fenômeno abre margem ao surgimento de um novo paradigma, denominado pela literatura de sensoriamento participativo, cuja ideia principal gira em torno de capacitar um cidadão comum a coletar e compartilhar dados de sensoriamento do ambiente em que está inserido, a partir de seu telefone celular \cite{kanhere}.

Ainda segundo Kanhere, sensoriamento participativo possui quatro vantagens principais sobre redes de sensores tradicionais (já que esta necessita de uma gama considerável de dispositivos sem fio, principalmente em áreas urbanizadas), sendo elas: custo de implementação baixo, já que usa telefones celulares e Wi-Fi; uma ampla mobilidade e cobertura proporcionada pelas operadoras de telefonia; economias de escala proporcionadas pelo uso de celulares; a facilidade assegurada pelas lojas de aplicativos e também as inúmeras formas de desenvolvimento de software disponíveis para sistemas operacionais de dispositivos móveis \cite{kanhere}. 

Com base em todos esses conceitos, estabelece-se, a seguir, um breve estado da arte registrado na literatura, no que se refere à área de desenvolvimento de aplicativos para plataformas móveis dentro da ideia de \emph{crowdsourcing} e \emph{smart cities}.

\section{Estado da Arte}

Conforme já relacionado na Introdução, \citeonline{sujatha} propuseram um sistema chamado MBTS para que qualquer pessoa obtenha informações de um determinado ônibus através de um aplicativo móvel. 

Três são os atores do sistema: os passageiros, os chamados ``coordenadores'' do ônibus e as pessoas que estão aguardando em um ponto. Todos os atores necessitam de um \textit{smartphone} com sistema operacional Android conectado à Internet. Continuamente, os passageiros e os ``coordenadores'' alimentam um banco de dados com informações sobre o ônibus. Essas informações podem ser as coordenadas do ônibus, obtidas através do GPS integrado ao \emph{smartphone}, ou até mesmo a ``situação'' do ônibus -- se ele está lotado ou vazio, se houve algum acidente, se o ônibus está atrasado, entre outros. Essas informações ficam armazenadas em um banco de dados para serem acessadas, posteriormente, por alguém que esteja utilizando o aplicativo, principalmente usuários que estão aguardando um ônibus em algum ponto. 

%Em linhas gerais, os objetivos do MBTS são os seguintes \cite{sujatha}:

%\begin{itemize}
%\item Utilização do GPS para obter a posição do ônibus e GSM para transmissão de informação.
%\item Obtenção da posição do ônibus (latitude e longitude) em diferentes intervalos de tempo.
%\item Transmissão da posição do veículo e outras informações (\emph{feedbacks}) para uma ``estação de monitoramento''  (o servidor que hospeda o banco de dados), depois de um intervalo de tempo especificado.
%\item Localizar a posição do veículo no Google Maps através de
%solicitação de rastreio, disponível pelo aplicativo móvel.
%\item Fornecer ao usuário final todas as informações relacionadas ao ônibus (\emph{feedbacks}), bem como a posição do veículo.
%\end{itemize}

A arquitetura do MBTS pode ser observada na Figura \ref{fig:archMBTS}. Basicamente, consiste de duas aplicações, uma Web e outra Android. A primeira permite o registro de usuários e ônibus e a segunda é utilizada para rastreio. A aplicação Web, desenvolvida em JSP, é voltada para o administrador do sistema e também funciona como um ``middleware'' para que a aplicação Android se conecte ao banco de dados e armazene informações \cite{sujatha}.

\begin{figure}[h]
\begin{center}
    \includegraphics[width=0.6\columnwidth]{../figs/arch_mbts.png}
    \caption{Arquitetura do MBTS.}Fonte: \cite{sujatha}
    \label{fig:archMBTS}
\end{center}
\end{figure}

\newpage

Uma outra pesquisa neste área de acompanhamento de ônibus, envolve a utilização de \sigla{RF}{Radio frequência} (Radio frequência). Foi desenvolvida por \citeonline{paradells}.

Os autores propõe a utilização de uma rede de dados composta por transmissores e receptores RF. Os transmissores situam-se nos ônibus, os quais transmitem informações para os receptores, que situam-se nos pontos de ônibus que, basicamente, são os nós da rede. Assim que um ônibus se aproxima de um ponto de ônibus - nó - inicia a transmissão de informações relevantes, captadas por sensores atrelados ao veículo. Essas informações podem ser usadas/acessadas por um usuário que está aguardando no referido ponto de ônibus. 

Uma limitação do sistema baseado em RF, e percebida pelos próprios autores, diz respeito à distância entre o ônibus e os nós da rede (neste caso os pontos de ônibus). Uma vez que sensores baseados em radio frequência possuem limitações de distância e na transmissão dos dados, acabam limitando a velocidade desenvolvida pelo veículo, o que torna o projeto inviável. 

Ainda, as informações só podem ser coletadas por usuários que já estão próximos ao veículo (ou em pontos de ônibus próximos ao veículo). Isso acaba tirando o propósito de acompanhamento ou rastreio de ônibus, pois o mesmo já está próximo do ponto, e pode ser vislumbrado por um usuário sem o auxílio de um sistema específico para tal.

\citeonline{alves} apresentaram um sistema -- chamado de \emph{Trip-planner} -- para planejar rotas em tempo real, para pessoas que utilizam transporte público em Lisboa. O sistema consegue informar quais são as melhores rotas e o tempo estimado de viagem para um determinado destino, baseados na estimativa de quantos veículos estão trafegando e quais são suas velocidades.

Informações de tempo-real são coletadas através do GPS equipado nos ônibus. Essas informações são utilizadas em um servidor (o que os autores chamam de \emph{Data Center}) para atualizar históricos e melhorar as estimativas, uma vez que é utilizado um algoritmo para predição de tempos de viagem. Esses históricos se referem à relatórios de quatro meses, com informações sobre tempos de viagem e velocidades dos veículos. Essas informações são analisadas e passam por um classificador, e uma vez processadas, são repassadas para qualquer dispositivo móvel conectados em uma rede sem fio, através de \emph{broadcast} \cite{alves}.

\newpage

A Figura \ref{fig:archLisbon} descreve de forma simplificada o sistema proposto por \citeonline{alves}. Nota-se claramente que dados históricos e de tempo-real são coletados com o auxílio do GPS instalado no ônibus. Esses dados passam pelo \emph{Data Center} e são utilizados como entrada para um algoritmo de predição. Uma vez processados, distribuem-se os dados para dispositivos móveis através de \emph{broadcast.}

\begin{figure}[h]
\begin{center}
    \includegraphics[width=0.85\columnwidth]{../figs/arch_lisbon.png}
    \caption{Arquitetura do \emph{Trip-planner}.}Fonte: \cite{alves}
    \label{fig:archLisbon}
\end{center}
\end{figure}
\chapter{Metodologia}\label{cap:metodologia}

A metodologia é uma descrição técnica de como será desenvolvido ou foi desenvolvido o trabalho. Devem estar detalhadas, de forma lógica, linear e cronológica, todas as etapas do projeto.

Deve ser explicado como o produto será gerado, quais são os principais fundamentos (algoritmos, paradigmas, teorias) e tecnologias (ambientes de desenvolvimento, linguagens de programação, plataformas de hardware) a serem empregados.

\section{Fundamentos}\label{s:fundamentos}
Indicar quais são os quais são os principais fundamentos (algoritmos, paradigmas, teorias) a serem empregados.
Cada fundamento utilizado deve ser justificado.

\section{Tecnologias}\label{s:tecnologias}
Indicar quais são os quais são as principais tecnologias (ambientes de desenvolvimento, linguagens de programação, plataformas de hardware) a serem empregadas. Cada tecnologia utilizada deve ser justificada.

\section{Como utilizar figuras}\label{s:usoFiguras}

A legenda de uma figura deve estar abaixo da mesma (ver Figura \ref{fig:ElemPreTextuais}).

\begin{figure}[h]
\begin{center}
    \includegraphics[width=0.5\columnwidth]{./ElemPreTextuais.png}
    \caption{Elementos Pré-Textuais de um Trabalho Acadêmico.}Fonte: \cite{NormasUTFPR}
    \label{fig:ElemPreTextuais}
\end{center}
\end{figure}

\section{Como utilizar tabelas}\label{s:usoTabelas}

O título de uma tabela deve estar acima da mesma. A Tabela \ref{tab:publicacao} ilustra um exemplo de uma tabela criada no latex\footnote{Se é a primeira vez que está trabalhando com o latex, comece acessando a página do Prof. Hugo Vieira em http://pessoal.utfpr.edu.br/hvieir/orient/}.

\begin{table}[h!]
%Se o nome da tabela for muito grande, é possível colocar entre colchetes o título reduzido que será mostrado na lista de tabelas e entre aspas o título completo que será mostrado acima da tabela. O mesmo é válido para figuras.

\caption[Distribuição do percentual de artigos segundo o ano de publicação]{Distribuição do percentual de artigos segundo o ano de publicação, período entre 1995 a 2005.}
\begin{center}
%\scalebox{0.75}{
\begin{tabular}{|c|c|c|c|c|c|}
\hline
\textbf{Ano de Publicação} &\textbf{Número de Artigos} & \textbf{\simbolo{\%}{Percentual} de Artigos} \\ \hline \hline
1995 & 1 & 2,6	\\
1998 & 3 &	7,9	\\
1999 & 3 &	7,9  \\
2000 & 2 & 5,2 \\
2001 & 5 & 13,2 \\
2002 & 5 &	13,2 \\
2003 & 5 &	13,2  \\
2004 & 7 &	18,4 \\
2005 & 7 & 18,4 \\
Total & 38 & 100 \\ \hline
\end{tabular}%}
\end{center}
\label{tab:publicacao}
\end{table}

\section{Como escrever equações}\label{s:Equações}

Veja a seguir exemplos equações criadas no latex.

\begin{equation}\label{eq:ACO1}
p_{ij}^{k}=\left\{\begin{matrix}
\frac{\tau _{ij}^{\alpha}(t)}{\sum_{j\in N_{i}^{k}}\tau_{ij}^{\alpha}(t)} \; \text{ se } j\in N_{i}^{k} \\
0 \; \text{ se } j\notin N_{i}^{k}
\end{matrix}\right.
\end{equation}

\begin{equation}\label{eq:ACO2}
\Delta \tau_{ij}^{k}(t) \propto ~\frac{1}{NSaltos^{k}(t)}.
\end{equation}

\begin{equation}\label{eq:ACO3}
\tau _{ij}(t+1) = \tau _{ij}(t)+\sum_{k=1}^{n_{k}}\Delta \tau _{i,j}^{k}(t).
\end{equation}

\begin{equation}\label{eq:ACO4}
\tau _{ij}(t+\Delta{t}) = (1-\rho)\tau_{ij}(t).
\end{equation}

\section{Como escrever algoritmos}\label{s:Algoritmos}

O Algoritmo \ref{alg:algS-ACO} apresenta o exemplo de um pseudo-código criado no latex. Para mudar o idioma dos comandos do algoritmo ou encontrar outros comandos, acesse o arquivo algorithmic.sty.

\begin{algorithm}
\caption{- Pseudocódigo do S-ACO \\ Fonte: Adaptado de \citeonline{ENGELBRECHT2007}}
\label{alg:algS-ACO}
\begin{algorithmic}[1]
\STATE Inicialize o tempo $t \leftarrow 0$
\STATE Inicialize o valor do feromônio $\tau_{i,j}(0)$
\STATE Crie $K$ formigas no nó de origem
\REPEAT
\FOR{cada formiga $k=1$ to $K$}
\STATE \COMMENT{Construindo um caminho $x^{k}(t)$}
\STATE $x^{k}(t) \leftarrow \emptyset$
\REPEAT
\STATE Selecione o próximo salto de acordo com a probabilidade da Equação~\ref{eq:ACO1}
\STATE Adicione o enlace $(i,j)$ ao caminho $x^{k}(t)$
\UNTIL {o destino seja alcançado}
\STATE Remova todos os \emph{loops} do caminho $x^{k}(t)$
\STATE Calcule a qualidade do caminho $f(x^{k}(t))$
\ENDFOR
\FORALL{enlace $(i,j)$ do grafo}
\STATE \COMMENT{Evaporação do Feromônio}
\STATE Reduza o feromônio $\tau_{i,j}(t)$ usando a Equação~\ref{eq:ACO4}
\ENDFOR
\FOR{formiga $k=1$ até $K$}
\FORALL{enlace $(i,j)$ de $x^{k}(t)$}
\STATE Atualize a quantidade de feromônio $\tau_{i,j}(t)$ usando as Equações~\ref{eq:ACO2} e~\ref{eq:ACO3}
\ENDFOR
\ENDFOR
\STATE $t \leftarrow t+1$
\UNTIL{que a condição de parada seja satisfeita}
\RETURN <caminho $x^{k}(t)$ de melhor qualidade como a solução>
\end{algorithmic}
\end{algorithm}
\chapter{Recursos de Hardware e Software}\label{cap:recursos}

Apresente aqui todos os recursos essenciais ao desenvolvimento do projeto. O(s) aluno(s) deve(m) mencionar também a origem dos recursos (próprios, externos ou da UTFPR) e a viabilidade do projeto.

\section{Recursos de Hardware}\label{s:hardware}

Indique aqui os recursos de hardware como componentes digitais, analógicos, fontes de alimentação, baterias, sensores, atuadores, entre outros. Especifique a origem dos recursos de hardware.

\section{Recursos de Software}\label{s:software}

Devem ser apresentados aqui os recursos de software, incluindo os principais fundamentos (teorias, algoritmos, paradigmas) e tecnologias (ambientes de desenvolvimento, linguagens de programação) a serem empregados. Especifique a origem dos recursos de software.
\chapter{Viabilidade e Cronograma Preliminar}\label{cap:viabilidade}

Neste capítulo serão apresentadas a viabilidade técnica e financeira do projeto, bem como o cronograma preliminar para o desenvolvimento do mesmo.
%Apresente aqui a viabilidade e o cronograma preliminar do projeto.

\section{Viabilidade}\label{s:viabilidade}

Uma vez que o projeto envolve programação em Java, é altamente viável tecnicamente, pois ambos os integrantes são familiarizados com esta linguagem de programação. Além disso, a aplicação derivada do projeto executará em \textit{smartphones} e \textit{tablets} com sistema operacional Android, muito popular e disseminado entre a população nos dias de hoje. Os próprios integrantes do projeto possuem tais dispositivos, o que facilita ainda mais o teste e a constatação da aplicação prática do projeto.

Além disso, o projeto é viável financeiramente. Não será necessária a compra de equipamentos e, uma vez que o projeto envolverá apenas \textit{software}, com a utilização de ferramentas gratuitas ou livres -- conforme já exposto neste documento -- os custos financeiros serão mínimos ou nulos.

%Relate a viabilidade técnica e financeira do projeto.
\newpage

\section{Cronograma Preliminar}\label{s:cronograma}

A Tabela \ref{tab:cronograma} fornece informações acerca das etapas necessárias para o desenvolvimento do projeto, bem como os seus tempos de duração estimados.

\begin{table}[h!]
%Se o nome da tabela for muito grande, é possível colocar entre colchetes o título reduzido que será mostrado na lista de tabelas e entre aspas o título completo que será mostrado acima da tabela. O mesmo é válido para figuras.

\caption[Cronograma preliminar]{Cronograma preliminar. Fonte: Autoria própria.}
\begin{center}
%\scalebox{0.75}{
\begin{tabular}{|>{\centering\arraybackslash}p{2cm}|>{\centering\arraybackslash}p{8.5cm}|>{\centering\arraybackslash}p{2cm}|}
%{|p{2.75cm}|p{7.75cm}|p{2.75cm}|}
\hline
\textbf{Número}     &\textbf{Etapa}      & \textbf{Dias} \\ \hline \hline
1 & Estudo de APIs e do Android SDK & 10 \\ \hline
2 & Projeto de software (modelagem com UML) & 10 \\ \hline
3 & Implementação do projeto de software & 30 \\ \hline
4 & Testes do protótipo utilizando um dispositivo móvel -- emulado -- com sistema operacional Android & 7 \\ \hline
5 & Testes do protótipo utilizando um dispositivo móvel -- real -- com sistema operacional Android & 7 \\ \hline
6 & Finalização do artefato & 14\\ \hline
7 & Finalização da documentação & 14\\ \hline
 & \raggedleft \textbf{Total} & 92\\ \hline
\end{tabular}%}
\end{center}
\label{tab:cronograma}
\end{table}
\chapter{Contexto}\label{cap:contexto}

Neste item deve ser explicitado quem é o cliente do projeto, se o projeto faz parte de um projeto maior, se ele é contratado por uma empresa externa a UTFPR e que se responsabilizará pelos recursos, entre outras possibilidades.
\chapter{Especificação: Projeto de Software}\label{cap:especificacao}

O presente capítulo apresenta os requisitos funcionais, não-funcionais e os diagramas de caso de uso para o projeto em questão.

\section{Requisitos funcionais}

Nesta seção serão apresentados os requisitos funcionais do aplicativo móvel.

\begin{description}
  \item[RF1] O aplicativo móvel deverá ser executado em um \textit{smartphone} ou \textit{tablet} com sistema operacional Android.
  %\item[RF2]  O aplicativo móvel deverá utilizar a Internet para seu correto funcionamento.
  %\subitem RF2.1 A conexão com a Internet deverá ser estabelecida de qualquer forma dentre as possíveis existentes, como uso de rede de dados 3G/4G ou WiFi.
  \item[RF2] O aplicativo móvel deverá coletar as coordenadas do dispositivo no qual está sendo executado.
  \item[RF3] O aplicativo móvel deverá apresentar ao usuário uma tela com as opções para acompanhamento de ônibus.
  \item[RF4] O aplicativo móvel deverá oferecer um recurso para que o usuário submeta \textit{feedbacks} sobre uma determinada linha de ônibus.
  \subitem RF4.1 Os \textit{feedbacks} deverão estar pré-definidos no aplicativo.
  \subitem RF4.2 Os \textit{feedbacks} devem conter informações relevantes sobre um determinado ônibus, como por exemplo, se o mesmo está lotado, se atrasou, ou se houve algum acidente no seu percurso.
  \item[RF5] O aplicativo móvel deverá armazenar as informações sobre uma determinada linha de ônibus.
  \subitem RF5.1 As informações deverão ficar armazenadas por, no máximo, 10 minutos.
  \subitem RF5.2 As informações podem ser provenientes do próprio usuário/dispositivo móvel ou de outros usuários/dispositivos móveis.
  \item[RF6] O aplicativo móvel deverá procurar por usuários/dispositivos próximos que estejam utilizando o aplicativo.
  \subitem RF6.1 Os dispositivos móveis devem estar com Wi-Fi ligado e executando o aplicativo móvel.
  \item[RF7] O aplicativo móvel deverá encaminhar suas informações armazenadas para quaisquer outros dispositivos móveis (\textit{smartphones} ou \textit{tablets}) encontrados.
  \subitem RF7.1 Os dispositivos móveis devem estar com Wi-Fi ligado e executando o aplicativo móvel.
  \item[RF8] O aplicativo móvel deverá oferecer ao usuário um recurso para mostrar um mapa informando a posição de um ônibus.
\end{description}

\section{Requisitos não-funcionais}

Nesta seção serão apresentados os requisitos não-funcionais do aplicativo móvel.

\begin{description}
  \item[RNF1] O aplicativo móvel deverá executar sobre o sistema operacional Android versão 4.1 (Jelly Bean), no mínimo.
  \item[RNF2]  O aplicativo móvel deverá utilizar a tecnologia Wi-Fi para seu correto funcionamento.
  \item[RNF3] O aplicativo móvel deverá utilizar uma arquitetura de rede descentralizada, integrando conceitos de DTN e P2P.
  \item[RNF4] O aplicativo móvel deverá ser desenvolvido na linguagem Java.
  \item[RNF5] O aplicativo móvel deverá ser desenvolvido utilizando o conjunto de ferramentas do Android SDK.
  \item[RNF6] O aplicativo móvel deverá ser desenvolvido utilizando o ambiente de desenvolvimento Android Studio.
\end{description}

\newpage

\section{Diagramas de Casos de Uso}

Esta seção apresenta o Diagrama de Casos de Uso para o aplicativo móvel a ser desenvolvido, bem como a descrição detalhada de cada caso. O diagrama pode ser observado na Figura \ref{fig:uc}.

\begin{figure}[h]
\begin{center}
    \includegraphics[width=1\columnwidth]{../figs/uc_tcc.png}
    \caption{Diagrama de Casos de Uso.}Fonte: Autoria própria.%\cite{sujatha}
    \label{fig:uc}
\end{center}
\end{figure}
%%%%%%%%%%%%%%%%%%%%%%%%%%%%%%%%%%%%%%%%%%%%%%%%%%%%%%%%%%%%%%%%%%%%%%%%%
\subsection{Coletar coordenada }

\noindent Nome: Coletar coordenada\\
Ator principal:Aplicativo móvel\\
Ator de bastidor: N/A\\
Descrição: quando se desejar conhecer a localização em que se encontra o dispositivo móvel, o GPS embutido no \textit{smartphone} será acionado para que as coordenadas sejam coletadas. \\\\\\\\
\textbf{Pré-condições:}
	\begin{itemize}
		\item O dispositivo móvel possui um sistema de GPS.
	\end{itemize}
\textbf{Pós-condições:}
	\begin{itemize}
		\item O GPS retorna a localização em que se encontra o dispositivo.
	\end{itemize}	
\textbf{Fluxo Básico:}
	\begin{enumerate}
		\item O Aplicativo móvel coleta a localização atual do dispositivo móvel, através do GPS integrado.
		\item As coordenadas obtidas pelo GPS são armazenadas na memória do \textit{smartphone}.
	\end{enumerate}	
\textbf{Fluxo Alternativo:}
	\begin{itemize}
		\item O Aplicativo móvel não consegue acionar o dispositivo de GPS: uma mensagem de erro é apresentada na tela do aparelho.
		\item O GPS é acionado, mas não é possível obter as coordenadas em que se encontra o dispositivo: uma mensagem de erro é apresentada na tela do aparelho.
		\item Não há espaço suficiente para armazenar as novas coordenadas obtidas do GPS: uma mensagem de erro é apresentada na tela do aparelho.
	\end{itemize}
\textbf{Regras de Negócio:}	nenhuma.
%%%%%%%%%%%%%%%%%%%%%%%%%%%%%%%%%%%%%%%%%%%%%%%%%%%%%%%%%%%%%%%%%%%%%%%%%
\subsection{Armazenar informações }

\noindent Nome: Armazenar informações\\
Ator principal:	Aplicativo móvel\\
Ator de bastidor: Usuário\\
Descrição: este caso de uso ocorre quando o usuário submete alguma informação referente à algum ônibus, quando as coordenadas são coletadas ou quando informações são recebidas de outros dispositivos.\\\\
\textbf{Pré-condições:}
	\begin{itemize}
		\item Necessário haver informações para armazenamento, sejam elas inseridas pelo próprio usuário, coletadas pelo GPS integrado ou recebidas de outros dispositivos.
	\end{itemize}
\textbf{Pós-condições:}
	\begin{itemize}
		\item Informação é armazenada na memória do dispositivo, por 10 minutos.
	\end{itemize}	
\textbf{Fluxo Básico:}
	\begin{enumerate}
		\item Após o Usuário submeter alguma informação, o aplicativo móvel armazena-a em memória por 10 minutos.
	\end{enumerate}	
\textbf{Fluxo Alternativo I:}
	\begin{enumerate}
		\item Informações são recebidas de outros dispositivos.
		\item Essas informações são armazenadas em memória por 10 minutos.
	\end{enumerate}
\textbf{Fluxo Alternativo II:}
	\begin{enumerate}
		\item A coordenada do dispositivo foi coletada.
		\item Armazena a coordenada em memória, por 10 minutos.
	\end{enumerate}
\textbf{Regras de Negócio:}	nenhuma.

%%%%%%%%%%%%%%%%%%%%%%%%%%%%%%%%%%%%%%%%%%%%%%%%%%%%%%%%%%%%%%%%%%%%%%%%%
\subsection{Enviar informações }

\noindent Nome: Enviar informações\\
Ator principal:	Aplicativo móvel\\
Ator de bastidor: Usuário\\
Descrição: este caso de uso ocorre quando há informações armazenadas no dispositivo, prontas para serem enviadas.\\\\
\textbf{Pré-condições:}
	\begin{itemize}
		\item Necessário haver informações para envio, sejam elas inseridas pelo próprio usuário, coletadas pelo GPS integrado ou recebidas de outros dispositivos. 
		\item Outros dispositivos móveis nas proximidades devem estar conectados e executando a aplicação.
	\end{itemize}
\textbf{Pós-condições:}
	\begin{itemize}
		\item Informações são enviadas para outros dispositivos móveis próximos que estejam conectados.
	\end{itemize}	
\textbf{Fluxo Básico:}
	\begin{enumerate}
		\item O Usuário decide notificar demais usuários sobre alguma informação relevante.
		\item O Usuário seleciona a informação no menu com opções pré-definidas.
		\item O Usuário confirma determinada informação para o ônibus em específico.
		\item O Aplicativo móvel envia a informação para quaisquer outros dispositivos móveis próximos que estejam conectados.
		\item A informação fica armazenada em memória por 10 minutos.
	\end{enumerate}	
\textbf{Fluxo Alternativo I:}
	\begin{itemize}
		\item Informações são recebidas de outros dispositivos móveis.
		\item As informações são repassadas para demais dispositivos móveis próximos que estejam conectados.
		\item Essas informações ficam armazenadas em memória por 10 minutos.
	\end{itemize}
\textbf{Fluxo Alternativo II:}
	\begin{enumerate}
		\item A coordenada do dispositivo foi coletada.
		\item A coordenada é repassada para demais dispositivos móveis próximos que estejam conectados.
		\item A coordenada fica armazenada em memória por 10 minutos.
	\end{enumerate}
\textbf{Regras de Negócio:}	nenhuma.

%%%%%%%%%%%%%%%%%%%%%%%%%%%%%%%%%%%%%%%%%%%%%%%%%%%%%%%%%%%%%%%%%%%%%%%%%
\subsection{Receber informações }

\noindent Nome: Receber informações\\
Ator principal:	Aplicativo móvel\\
Ator de bastidor: N/A\\
Descrição: este caso de uso ocorre quando o aplicativo móvel recebe informações/mensagens de outros dispositivos.\\\\
\textbf{Pré-condições:}
	\begin{itemize}
		\item As mensagens recebidas devem estar construídas de acordo com um protocolo pré-configurado no dispositivo.
		\end{itemize}
\textbf{Pós-condições:}
	\begin{itemize}
		\item As informações/mensagens são armazenadas na memória por 10 minutos.
	\end{itemize}	
\textbf{Fluxo Básico:}
	\begin{enumerate}
		\item O aplicativo móvel recebe uma informação/mensagem de outro dispositivo móvel.
		\item O aplicativo móvel verifica se reconhece o padrão da mensagem recebida.
		\item O aplicativo móvel armazena a nova mensagem na memória de mensagens recebidas (para consulta local).
		\item O aplicativo móvel armazena a nova mensagem na memória de mensagens a enviar (para propagar a mensagem a outros dispositivos).
	\end{enumerate}	
\textbf{Fluxo Alternativo:}
	\begin{itemize}
		\item A mensagem recebida não possui os padrões esperados e é automaticamente descartada.
		\item A memória destinada a alocação de mensagens recebidas está cheia: uma mensagem de notificação é mostrada na tela do dispositivo.
	\end{itemize}
\textbf{Regras de Negócio:}	nenhuma.

%%%%%%%%%%%%%%%%%%%%%%%%%%%%%%%%%%%%%%%%%%%%%%%%%%%%%%%%%%%%%%%%%%%%%%%%%
\subsection{Procurar por dispositivos }

\noindent Nome: Procurar por dispositivos\\
Ator principal:	Aplicativo móvel\\
Ator de bastidor: Usuário\\
Descrição: este caso de uso ocorre periodicamente durante a execução do aplicativo móvel.\\\\
\textbf{Pré-condições:}
	\begin{itemize}
		\item O aplicativo móvel deve estar executando.
		\item O Wi-Fi no dispositivo móvel deve estar ligado.
	\end{itemize}
\textbf{Pós-condições:}
	\begin{itemize}
		\item Uma conexão é estabelecida com outros dispositivos móveis que estejam próximos e executado o aplicativo.
	\end{itemize}	
\textbf{Fluxo Básico:}
	\begin{enumerate}
		\item O Usuário liga a conexão Wi-Fi no dispositivo móvel.
		\item O Usuário inicia o Aplicativo móvel.
		\item O Aplicativo móvel procura por outros dispositivos móveis nas proximidades.
		\item Se um dispositivo é encontrado, inclui o mesmo na lista de dispositivos encontrados e estabelece conexão.
		\item O Aplicativo móvel aguarda dois minutos.
		\item Realiza nova busca, retornando ao passo 3.
	\end{enumerate}	
\textbf{Fluxo Alternativo:}
	\begin{itemize}
		\item O usuário pressiona o botão para procurar por dispositivos imediatamente.
		\item Vai para o passo 3 do Fluxo Básico.
	\end{itemize}
\textbf{Regras de Negócio:}	nenhuma.


%
%   ADICIONAR REFERÊNCIAS NESTE CAPÍTULO
%
\chapter{Gestão e planejamento}\label{cap:gestao}

O total de horas estimadas para desenvolvimento do projeto foi calculada através do método de \textit{Use Case Points}, que baseia-se em informações extraídas diretamente dos diagramas de Casos de Uso especificados.

O presente capítulo mostra todos os passos desenvolvimentos para o cálculo dos \textit{Use Case Points}, bem como a utilização dos métodos \sigla{PERT}{\textit{Program Evaluation and Review Technique}} (\textit{Program Evaluation and Review Technique}) e \sigla{CPM}{\textit{Critical Path Method}} (\textit{Critical Path Method}) para elaboração da Rede PERT-CPM, pela qual é possível observar o caminho crítico para o desenvolvimento do projeto, usando como base as tarefas e suas horas estipuladas.

\section{\textit{Use Case Points}}

A estimativa por \textit{Use Case Points} consiste em alguns passos. \citeonline{clemmons} exemplificou os passos para o cálculo do \textit{Use Case Points}, os quais serão seguidos neste capítulo.

O primeiro passo é calcular o \sigla{UAW}{\textit{Unadjusted Actor Weight}} (\textit{Unadjusted Actor Weight}), o peso total dos atores do sistema. Os atores foram classificados como: ator complexo (Usuário), com peso 3 e ator médio (Aplicativo móvel), com peso 2.

\begin{center}
UAW = 3x1 + 2x1 = 5
\end{center}

Em seguida, calcula-se o peso não-ajustado de cada Caso de Uso do sistema. Os casos de uso foram classificados da seguinte forma:

\begin{itemize}
\item Coletar coordenada: caso de uso médio (peso 10).
\item Armazenar informações: caso de uso médio (peso 10).
\item Enviar informações: caso de uso complexo (peso 15).
\item Receber informações: caso de uso médio (peso 10).
\item Procurar por dispositivos: caso de uso médio (peso 10).
\end{itemize}

O \sigla{UUCW}{\textit{Unadjusted Use Case Weight}} (\textit{Unadjusted Use Case Weight}) é então calculado:

\begin{center}
UUCW = 10x3 + 15x2 = 60
\end{center}

No terceiro passo, calcula-se o \sigla{UUCP}{\textit{Unadjusted Use Case Point}} (\textit{Unadjusted Use Case Point}), que consiste na soma de UAW com UUCW:

\begin{center}
UUCP = UAW + UUCW = 65
\end{center}

Prossegue-se então para o cálculo de fatores técnicos, que cobre requisitos funcionais do sistema, e fatores de ambiente, que cobre requisitos não-funcionais associados ao processo de desenvolvimento.

A variável de fatores técnicos, TFactor, é obtida através do somatório dos níves F1 a F13, multiplicados pelo seu peso:

\begin{table}[h!]
\caption[Fatores técnicos]{Fatores técnicos. Fonte: Autoria própria.}
\begin{center}
%\scalebox{0.75}{
\begin{tabular}{|c|c|c|c|c|}
\hline
\textbf{Fator} & \textbf{Fatores que contribuem para a complexidade} & \textbf{Peso} & \textbf{Valor}  & \textbf{Total} \\ \hline \hline
F1  & Sistemas distribuídos & 2 & 5 & 10	\\
\hline
F2  & Tempo de resposta & 1 & 5 & 5	\\
\hline
F3  & Eficiência para o usuário final (on-line) & 1 & 5 & 5	\\
\hline
F4  & Processamento interno complexo & 1 & 4 & 4	\\
\hline
F5  & Código reusável & 1 & 2 & 2	\\
\hline
F6  & Facilidade de instalação & 0,5 & 1 & 0,5	\\
\hline
F7  & Facilidade de uso (facilidade operacional) & 0,5 & 4 & 2	\\
\hline
F8  & Portabilidade & 2 & 3 & 6	\\
\hline
F9  & Facilidade de mudança & 1 & 3 & 3	\\
\hline
F10  & Concorrência (acesso simultâneo à aplicação) & 1 & 4 & 4	\\
\hline
F11  & Recursos de segurança & 1 & 3 & 3	\\
\hline
F12  & Fornece acesso direto para terceiros & 1 & 0 & 0	\\
\hline
F13  & Requer treinamento especial para o usuário & 1 & 0 & 0	\\
\hline
TFactor & & & & 44,5	\\
\hline
\end{tabular}%}
\end{center}
\label{tab:tfactor}
\end{table}

O \sigla{TCF}{\textit{Technical Complexity Factor}} (\textit{Technical Complexity Factor}) é então calculado:

\begin{center}
TCF = 0,6 + (0,01 x TFactor) = 0,6 + (0,01 x 44,5) = 1,045
\end{center}

\newpage

A variável de fatores de ambiente, EFactor, é calculada através do somatório dos níveis F1 a F8, multiplicados pelo seu peso:

\begin{table}[h!]
\caption[Fatores de ambiente]{Fatores de ambiente. Fonte: Autoria própria.}
\begin{center}
%\scalebox{0.75}{
\begin{tabular}{|c|m{9cm}|c|c|c|}
\hline
\textbf{Fator} & \textbf{Fatores que contribuem para a complexidade} & \textbf{Peso} & \textbf{Valor}  & \textbf{Total} \\ \hline \hline
F1  & Familiaridade da equipe com o processo formal de desenvolvimento adotado & 1,5 & 4 & 6	\\
\hline
F2  & Colaboradores de meio período & -1 & 2 & -2	\\
\hline
F3  & Capacidade do líder de projeto em análise de requisitos e modelagem & 0,5 & 4 & 2	\\
\hline
F4  & Experiência da equipe em desenvolvimento de aplicações do gênero em questão & 0,5 & 5 & 2,5	\\
\hline
F5  & Experiência em orientação a objetos & 1 & 5 & 5	\\
\hline
F6  & Motivação da equipe & 1 & 5 & 5	\\
\hline
F7  & Dificuldades com a linguagem de programação & -1 & 3 & -3	\\
\hline
F8  & Requisitos estáveis & 2 & 3 & 6	\\
\hline
EFactor & & & & 21,5	\\
\hline
\end{tabular}%}
\end{center}
\label{tab:efactor}
\end{table}

O \sigla{EF}{\textit{Environmental Factor}} (\textit{Environmental Factor}) é então calculado:

\begin{center}
EF = 1,4 + (-0,03 x EFactor) = 1,4 + (-0,03 x 21,5) = 0,755
\end{center}

Com os resultados obtidos nos passos anteriores, O valor de \sigla{UCP}{\textit{Use Case Points}} (\textit{Use Case Points}) pode ser então calculado:

\begin{center}
UCP = UUCP x TCF x EF = 51,28
\end{center}

O tempo de trabalho estimado pode ser obtido multiplicando-se UCP por 20. Portanto, estimou-se 1025,6 horas de trabalho.

\newpage

\section{Gerenciamento de tempo}

Para a elaboração da Rede PERT-CPM foi necessário, primeiramente,
levantar as tarefas do projeto em questão. As tarefas levantadas, bem como a duração de cada uma delas podem ser observadas na Tabela  \ref{tab:tarefas}:

\begin{table}[h!]
\caption[Tarefas do projeto]{Tarefas do projeto. Fonte: Autoria própria.}
\begin{center}
%\scalebox{0.75}{
\begin{tabular}{|c|m{10cm}|c|}\hline
\Centering\bfseries \textbf{Número} & \Centering\bfseries \textbf{Tarefa} & \Centering\bfseries \textbf{Duração}
%\textbf{Número} & \textbf{Tarefa} & \textbf{Duração} 
\\ \hline \hline
1 & Estudo de APIs e do Android SDK & 100	\\
\hline
2 & Projeto de software (modelagem UML) & 100	\\
\hline
3 & Implementação do projeto de software & 280	\\
\hline
4 & Testes do protótipo utilizando um dispositivo móvel
(emulado) com sistema operacional Android & 60	\\
\hline
5 & Testes do protótipo utilizando um dispositivo móvel (real)
com sistema operacional Android & 100	\\
\hline
6 & Finalização do artefato  & 192	\\
\hline
7 & Testes e ajustes finais no software  & 104	\\
\hline
8 & Finalização da documentação & 80	\\
\hline
9 & Preparação da apresentação para banca  & 8	\\
\hline
Total & & 1024	\\
\hline
\end{tabular}%}
\end{center}
\label{tab:tarefas}
\end{table}

Utilizando-se dos resultados expostos, a Rede PERT-CPM para o projeto é mostrada na Figura \ref{fig:pertcpm}.

\begin{figure}[h]
\begin{center}
    \includegraphics[width=1\columnwidth]{../figs/pert-cpm.png}
    \caption{Rede PERT-CPM.}Fonte: Autoria própria.%\cite{sujatha}
    \label{fig:pertcpm}
\end{center}
\end{figure}

Na figura, o ``cedo" de cada evento é evidenciado na cor azul, ao passo que o ``tarde" de cada evento é evidenciado na cor vermelha. A folga, diferença entre tarde e cedo, é apresentada na cor laranja/amarelo.

O caminho crítico fica destacado pelas setas em vermelho, e perpassa pelos eventos 1, 2, 3, 4, 5, 7, 8 e 10 e, por consequência, pelas tarefas 1, 2, 3, 4, 6, 7 e 9.
\chapter{Análise de riscos}\label{cap:analiseriscos}

O presente capítulo apresenta a análise de riscos elaborada para o projeto. A análise consiste, basicamente, em seis riscos, que são:

\begin{enumerate}
\item Escolha de tecnologias que não atendam às necessidades do projeto.
\item Dificuldades no desenvolvimento do software.
\item Descumprimento do prazo estabelecido para a realização de alguma etapa prevista no cronograma do projeto.
\item Definição imprecisa do escopo do projeto.
\item Desistência de um membro da equipe.
\item Inexistência de tempo para a conclusão do projeto.
\end{enumerate}

Na sequência serão apresentadas a descrição, avaliação do risco, impacto, probabilidade e estratégia de ação correspondentes a cada um dos riscos levantados. A avaliação do risco e a probabilidade são classificadas na seguinte escala: baixa, baixa-média, média, média-alta e alta.

\begin{table}[h!]
\caption[Análise do risco 1]{Análise do risco 1. Fonte: Autoria própria.}
\begin{center}
%\scalebox{0.75}{
\begin{tabular}{|c|p{11cm}|}
\hline
\textbf{Risco} & Escolha de tecnologias que não atendam às necessidades do projeto. \\
\hline
\textbf{Descrição} & O projeto envolve o uso de várias tecnologias, portanto, há a possibilidade de escolha equivocada de alguma delas o que pode vir a não satisfazer os requisitos mínimos necessários. \\
\hline
\textbf{Avaliação do risco} & Médio-Alto. \\
\hline
\textbf{Impacto} & Caso o emprego de uma tecnologia seja incapaz de suprir as necessidades mínimas requeridas, a etapa que faz uso deste não poderá ser finalizada. \\ 
\hline
\textbf{Probabilidade} & Média. \\
\hline
\textbf{Estratégia de ação} & As tecnologias que serão utilizados terão seus requisitos detalhadamente avaliados para que o grupo possa eliminar escolhas equivocadas.	\\
\hline
\end{tabular}%}
\end{center}
\label{tab:risco1}
\end{table}

\begin{table}[h!]
\caption[Análise do risco 2]{Análise do risco 2. Fonte: Autoria própria.}
\begin{center}
%\scalebox{0.75}{
\begin{tabular}{|c|p{11cm}|}
\hline
\textbf{Risco} & Dificuldades no desenvolvimento do software. \\
\hline
\textbf{Descrição} & Apesar de um médio conhecimento das tecnologias, corre-se o risco de empecilhos aparecem mesmo com esse conhecimento prévio. Deve-se observar ainda que o trabalho propõem-se a desenvolver uma nova arquitetura de rede que poderá se mostrar impraticável de implementação.\\
\hline
\textbf{Avaliação do risco} & Alto. \\
\hline
\textbf{Impacto} & Sendo o desenvolvimento do aplicativo o cerne do projeto, problemas com o mesmo terão que ser cuidadosamente analisados e solucionados para não inviabilizarem o projeto. \\ 
\hline
\textbf{Probabilidade} & Média-Alta. \\
\hline
\textbf{Estratégia de ação} &  Antes de se iniciar a codificação, a equipe fará a modelagem do software para que se tenha claramente definido o comportamento do mesmo, facilitando testes e depuração de código.	\\
\hline
\end{tabular}%}
\end{center}
\label{tab:risco1}
\end{table}

\begin{table}[h!]
\caption[Análise do risco 3]{Análise do risco 3. Fonte: Autoria própria.}
\begin{center}
%\scalebox{0.75}{
\begin{tabular}{|c|p{11cm}|}
\hline
\textbf{Risco} & Descumprimento do prazo estabelecido para a realização de alguma etapa prevista no cronograma do projeto. \\
\hline
\textbf{Descrição} & O projeto será dividido em etapas e estas serão temporizadas de acordo com o seu grau de dificuldade para então designar-se um membro da equipe para realizá-la. Caso este membro não a conclua no prazo estipulado, etapas dependentes desta terão que ser adiadas, prejudicando o projeto como um todo.\\
\hline
\textbf{Avaliação do risco} & Alto. \\
\hline
\textbf{Impacto} & Embora algumas tarefas possam ser realizadas paralelamente, existem aquelas que são dependentes de outras e, caso ocorra atraso em uma, a tarefa dependente terá que esperar até o cumprimento da outra, o que por sua vez pode atrasar outras tarefas, ocasionando um efeito em cadeia que pode prejudicar fortemente o desenvolvimento do projeto. \\ 
\hline
\textbf{Probabilidade} & Média.\\
\hline
\textbf{Estratégia de ação} &  Os membros da equipe terão reuniões semanais para verificar o progresso de cada etapa definida, pretende-se realizar as tarefas em grupo e dessa forma, facilitar o entendimento de uma tarefa que não esteja com um progresso adequado.	\\
\hline
\end{tabular}%}
\end{center}
\label{tab:risco1}
\end{table}

\begin{table}[h!]
\caption[Análise do risco 4]{Análise do risco 4. Fonte: Autoria própria.}
\begin{center}
%\scalebox{0.75}{
\begin{tabular}{|c|p{11cm}|}
\hline
\textbf{Risco} &  Definição imprecisa do escopo do projeto. \\
\hline
\textbf{Descrição} & Uma definição equivocada do escopo da proposta pode trazer grandes dificuldades no desenvolvimento do projeto, culminando na entrega incompleta do mesmo.\\
\hline
\textbf{Avaliação do risco} & Médio-Alto. \\
\hline
\textbf{Impacto} & Como em todo planejamento de projeto, procura-se deixar o escopo do mesmo o mais claro possível para evitar o não cumprimento dos objetivos. Dessa forma, o cumprimento das metas  é de fundamental importância seja em ambiente acadêmico ou empresarial. Logo, a descoberta de uma indefinição nesse aspecto colocaria em risco todo o planejamento e execução das tarefas. \\ 
\hline
\textbf{Probabilidade} & Média.\\
\hline
\textbf{Estratégia de ação} & Durante o planejamento do projeto, buscou-se definir seu escopo de tal forma que os membros da equipe já identificassem as principais áreas técnicas para o seu desenvolvimento. Assim, evita-se dispêndio de tempo para adquirir conhecimentos não pertinentes ao projeto.\\
\hline
\end{tabular}%}
\end{center}
\label{tab:risco1}
\end{table}

\begin{table}[h!]
\caption[Análise do risco 5]{Análise do risco 5. Fonte: Autoria própria.}
\begin{center}
%\scalebox{0.75}{
\begin{tabular}{|c|p{11cm}|}
\hline
\textbf{Risco} &  Desistência de um membro da equipe. \\
\hline
\textbf{Descrição} & Ao longo do projeto, há a possibilidade de que algum membro da equipe não se identifique mais com o projeto e queira abandoná-lo ou que seja obrigado, por motivos de força maior, a se afastar do projeto.\\
\hline
\textbf{Avaliação do risco} & Alto. \\
\hline
\textbf{Impacto} & Com a redução de integrantes da equipe, haveria uma sobrecarga sobre o membro remanescente. \\ 
\hline
\textbf{Probabilidade} & Baixa-Média.\\
\hline
\textbf{Estratégia de ação} & Na fase inicial do projeto todos os membros foram instigados à expor suas ideias de tal forma que o projeto como um todo tivesse a participação da dupla que o compõe, desta forma, aumentando a motivação individual dos integrantes para a completa realização do projeto. Todavia, se de fato essa situação ocorrer haverá a necessidade de uma reestruturação do plano de projeto.	\\
\hline
\end{tabular}%}
\end{center}
\label{tab:risco1}
\end{table}

\begin{table}[h!]
\caption[Análise do risco 6]{Análise do risco 6. Fonte: Autoria própria.}
\begin{center}
%\scalebox{0.75}{
\begin{tabular}{|c|p{11cm}|}
\hline
\textbf{Risco} &  Inexistência de tempo para a conclusão do projeto. \\
\hline
\textbf{Descrição} & Como os integrantes da equipe tem que fazer outas atividades em paralelo, pode faltar tempo para terminar o projeto em sua plenitude.\\
\hline
\textbf{Avaliação do risco} & Alto. \\
\hline
\textbf{Impacto} & Caso haja escassez de tempo será inviável o término do projeto dentro do prazo estipulado, ocasionado atraso no dia de sua apresentação.\\ 
\hline
\textbf{Probabilidade} & Baixa-Média.\\
\hline
\textbf{Estratégia de ação} & Estimar prazos para cada tarefa, e se necessário, reduzir os requisitos finais, diminuindo assim o tempo do projeto.	\\
\hline
\end{tabular}%}
\end{center}
\label{tab:risco1}
\end{table}
\input{conclusao}
%%---------- Referencias ----------

\chapter{Referências Bibliográficas}

Todas as referências citadas no texto devem estar relacionadas em ordem alfabética. \textbf{Ao utilizar o Latex, as referências são ordenadas automaticamente.}

Referências não citadas no texto do documento, não devem ser apresentadas na lista de Referências Bibliográficas.

As normas da ABNT para referências bibliográficas podem ser consultadas em \cite{NormasUTFPR}.

Para citar referências no texto utilizando o latex, use o comando $citeonline$ ou $cite$, precedido de uma barra invertida.

Com o comando $citeonline\{vendramin2012\}$, a impressão no texto ficará da seguinte forma:

Vendramin (2012).

Ou seja, a citação aparece inserida dentro do texto, como por exemplo:

Conforme a autora Vendramin (2012), a proposta de TCC ...

Ao usar o comando $cite\{vendramin2012\}$, a impressão no texto ficará da seguinte forma:

A proposta de TCC ... (Vendramin, 2012).

A seguir são apresentados exemplos de referências bibliográficas criadas no Latex. As referências são inseridas em um arquivo .bib e o nome desse arquivo (por exemplo, Referencias) deve ser usado ao chamar o comando $bibliography\{Referencias\}$ precedido por uma barra invertida. O comando $bibliography$ é responsável por gerar de forma automática, a partir do arquivo $Referencias.bib$, as referências bibliográficas citadas no documento.

\textbf{Livros}

Para exemplos de como referenciar livros veja as seguintes referências: \cite{castro2006} e \cite{ENGELBRECHT2007}.

\textbf{Artigos de Periódicos}

Para exemplos de como referenciar artigos de periódicos veja as seguintes referências: \cite{vendramin2012a} e \cite{cao2012}.

\textbf{Relatórios Técnicos e Normas Técnicas}

Para exemplos de como referenciar relatórios técnicos ou normas técnicas veja as seguintes referências: \cite{mills2004} e \cite{RFC5050}.

\textbf{Artigos Científicos em Congressos/Conferências/Simpósios}

Para exemplos de como referenciar artigos científicos veja as seguintes referências: \cite{vendramin2012b} e \cite{vendramin2011}.

\textbf{Monografias/Dissertações/Teses}

Para exemplos de como referenciar monografias, dissertações ou teses veja as seguintes referências: \cite{vendramin2012c} e \cite{saleem2001}.

\bibliography{referencias}

%---------- Apêndices e Anexos ----------
% Apêndice
%\appendix
%\chapter{Título do Apêndice}

Elemento opcional, que consiste em texto ou documento \textbf{elaborado pelo autor}, a fim de complementar sua argumentação, sem prejuízo da unidade nuclear do trabalho.

Os apêndices devem ser identificados por letras maiúsculas consecutivas, seguidas de travessão e respectivo título.

% Anexo
%\appendix
%\renewcommand{\appendixname}{Anexo} % Para criar Anexo ao invés de Apêndice
%\chapter{Título do Anexo}

Elemento opcional, que consiste em texto ou documento \textbf{não elaborado pelo autor}, que serve de fundamentação, comprovação e ilustração.

Os anexos devem ser identificados por letras maiúsculas consecutivas, seguidas de travessão e respectivo título.

\end{document}