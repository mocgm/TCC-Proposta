\chapter{Análise de riscos}\label{cap:analiseriscos}

O presente capítulo apresenta a análise de riscos elaborada para o projeto. A análise consiste, basicamente, em seis riscos, que são:

\begin{enumerate}
\item Escolha de tecnologias que não atendam às necessidades do projeto.
\item Dificuldades no desenvolvimento do software.
\item Descumprimento do prazo estabelecido para a realização de alguma etapa prevista no cronograma do projeto.
\item Definição imprecisa do escopo do projeto.
\item Desistência de um membro da equipe.
\item Inexistência de tempo para a conclusão do projeto.
\end{enumerate}

Na sequência serão apresentadas a descrição, avaliação do risco, impacto, probabilidade e estratégia de ação correspondentes a cada um dos riscos levantados. A avaliação do risco e a probabilidade são classificadas na seguinte escala: baixa, baixa-média, média, média-alta e alta.

\begin{table}[h!]
\caption[Análise do risco 1]{Análise do risco 1. Fonte: Autoria própria.}
\begin{center}
%\scalebox{0.75}{
\begin{tabular}{|c|p{11cm}|}
\hline
\textbf{Risco} & Escolha de tecnologias que não atendam às necessidades do projeto. \\
\hline
\textbf{Descrição} & O projeto envolve o uso de várias tecnologias, portanto, há a possibilidade de escolha equivocada de alguma delas o que pode vir a não satisfazer os requisitos mínimos necessários. \\
\hline
\textbf{Avaliação do risco} & Médio-Alto. \\
\hline
\textbf{Impacto} & Caso o emprego de uma tecnologia seja incapaz de suprir as necessidades mínimas requeridas, a etapa que faz uso deste não poderá ser finalizada. \\ 
\hline
\textbf{Probabilidade} & Média. \\
\hline
\textbf{Estratégia de ação} & As tecnologias que serão utilizados terão seus requisitos detalhadamente avaliados para que o grupo possa eliminar escolhas equivocadas.	\\
\hline
\end{tabular}%}
\end{center}
\label{tab:risco1}
\end{table}

\begin{table}[h!]
\caption[Análise do risco 2]{Análise do risco 2. Fonte: Autoria própria.}
\begin{center}
%\scalebox{0.75}{
\begin{tabular}{|c|p{11cm}|}
\hline
\textbf{Risco} & Dificuldades no desenvolvimento do software. \\
\hline
\textbf{Descrição} & Apesar de um médio conhecimento das tecnologias, corre-se o risco de empecilhos aparecem mesmo com esse conhecimento prévio. Deve-se observar ainda que o trabalho propõem-se a desenvolver uma nova arquitetura de rede que poderá se mostrar impraticável de implementação.\\
\hline
\textbf{Avaliação do risco} & Alto. \\
\hline
\textbf{Impacto} & Sendo o desenvolvimento do aplicativo o cerne do projeto, problemas com o mesmo terão que ser cuidadosamente analisados e solucionados para não inviabilizarem o projeto. \\ 
\hline
\textbf{Probabilidade} & Média-Alta. \\
\hline
\textbf{Estratégia de ação} &  Antes de se iniciar a codificação, a equipe fará a modelagem do software para que se tenha claramente definido o comportamento do mesmo, facilitando testes e depuração de código.	\\
\hline
\end{tabular}%}
\end{center}
\label{tab:risco1}
\end{table}

\begin{table}[h!]
\caption[Análise do risco 3]{Análise do risco 3. Fonte: Autoria própria.}
\begin{center}
%\scalebox{0.75}{
\begin{tabular}{|c|p{11cm}|}
\hline
\textbf{Risco} & Descumprimento do prazo estabelecido para a realização de alguma etapa prevista no cronograma do projeto. \\
\hline
\textbf{Descrição} & O projeto será dividido em etapas e estas serão temporizadas de acordo com o seu grau de dificuldade para então designar-se um membro da equipe para realizá-la. Caso este membro não a conclua no prazo estipulado, etapas dependentes desta terão que ser adiadas, prejudicando o projeto como um todo.\\
\hline
\textbf{Avaliação do risco} & Alto. \\
\hline
\textbf{Impacto} & Embora algumas tarefas possam ser realizadas paralelamente, existem aquelas que são dependentes de outras e, caso ocorra atraso em uma, a tarefa dependente terá que esperar até o cumprimento da outra, o que por sua vez pode atrasar outras tarefas, ocasionando um efeito em cadeia que pode prejudicar fortemente o desenvolvimento do projeto. \\ 
\hline
\textbf{Probabilidade} & Média.\\
\hline
\textbf{Estratégia de ação} &  Os membros da equipe terão reuniões semanais para verificar o progresso de cada etapa definida, pretende-se realizar as tarefas em grupo e dessa forma, facilitar o entendimento de uma tarefa que não esteja com um progresso adequado.	\\
\hline
\end{tabular}%}
\end{center}
\label{tab:risco1}
\end{table}

\begin{table}[h!]
\caption[Análise do risco 4]{Análise do risco 4. Fonte: Autoria própria.}
\begin{center}
%\scalebox{0.75}{
\begin{tabular}{|c|p{11cm}|}
\hline
\textbf{Risco} &  Definição imprecisa do escopo do projeto. \\
\hline
\textbf{Descrição} & Uma definição equivocada do escopo da proposta pode trazer grandes dificuldades no desenvolvimento do projeto, culminando na entrega incompleta do mesmo.\\
\hline
\textbf{Avaliação do risco} & Médio-Alto. \\
\hline
\textbf{Impacto} & Como em todo planejamento de projeto, procura-se deixar o escopo do mesmo o mais claro possível para evitar o não cumprimento dos objetivos. Dessa forma, o cumprimento das metas  é de fundamental importância seja em ambiente acadêmico ou empresarial. Logo, a descoberta de uma indefinição nesse aspecto colocaria em risco todo o planejamento e execução das tarefas. \\ 
\hline
\textbf{Probabilidade} & Média.\\
\hline
\textbf{Estratégia de ação} & Durante o planejamento do projeto, buscou-se definir seu escopo de tal forma que os membros da equipe já identificassem as principais áreas técnicas para o seu desenvolvimento. Assim, evita-se dispêndio de tempo para adquirir conhecimentos não pertinentes ao projeto.\\
\hline
\end{tabular}%}
\end{center}
\label{tab:risco1}
\end{table}

\begin{table}[h!]
\caption[Análise do risco 5]{Análise do risco 5. Fonte: Autoria própria.}
\begin{center}
%\scalebox{0.75}{
\begin{tabular}{|c|p{11cm}|}
\hline
\textbf{Risco} &  Desistência de um membro da equipe. \\
\hline
\textbf{Descrição} & Ao longo do projeto, há a possibilidade de que algum membro da equipe não se identifique mais com o projeto e queira abandoná-lo ou que seja obrigado, por motivos de força maior, a se afastar do projeto.\\
\hline
\textbf{Avaliação do risco} & Alto. \\
\hline
\textbf{Impacto} & Com a redução de integrantes da equipe, haveria uma sobrecarga sobre o membro remanescente. \\ 
\hline
\textbf{Probabilidade} & Baixa-Média.\\
\hline
\textbf{Estratégia de ação} & Na fase inicial do projeto todos os membros foram instigados à expor suas ideias de tal forma que o projeto como um todo tivesse a participação da dupla que o compõe, desta forma, aumentando a motivação individual dos integrantes para a completa realização do projeto. Todavia, se de fato essa situação ocorrer haverá a necessidade de uma reestruturação do plano de projeto.	\\
\hline
\end{tabular}%}
\end{center}
\label{tab:risco1}
\end{table}

\begin{table}[h!]
\caption[Análise do risco 6]{Análise do risco 6. Fonte: Autoria própria.}
\begin{center}
%\scalebox{0.75}{
\begin{tabular}{|c|p{11cm}|}
\hline
\textbf{Risco} &  Inexistência de tempo para a conclusão do projeto. \\
\hline
\textbf{Descrição} & Como os integrantes da equipe tem que fazer outas atividades em paralelo, pode faltar tempo para terminar o projeto em sua plenitude.\\
\hline
\textbf{Avaliação do risco} & Alto. \\
\hline
\textbf{Impacto} & Caso haja escassez de tempo será inviável o término do projeto dentro do prazo estipulado, ocasionado atraso no dia de sua apresentação.\\ 
\hline
\textbf{Probabilidade} & Baixa-Média.\\
\hline
\textbf{Estratégia de ação} & Estimar prazos para cada tarefa, e se necessário, reduzir os requisitos finais, diminuindo assim o tempo do projeto.	\\
\hline
\end{tabular}%}
\end{center}
\label{tab:risco1}
\end{table}