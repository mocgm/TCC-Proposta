\chapter{Viabilidade e Cronograma Preliminar}\label{cap:viabilidade}

Neste capítulo serão apresentadas a viabilidade técnica e financeira do projeto, bem como o cronograma preliminar para o desenvolvimento do mesmo.
%Apresente aqui a viabilidade e o cronograma preliminar do projeto.

\section{Viabilidade}\label{s:viabilidade}

Uma vez que o projeto envolve programação em Java, é altamente viável tecnicamente, pois ambos os integrantes são familiarizados com esta linguagem de programação. Além disso, a aplicação derivada do projeto executará em \textit{smartphones} e \textit{tablets} com sistema operacional Android, muito popular e disseminado entre a população nos dias de hoje. Os próprios integrantes do projeto possuem tais dispositivos, o que facilita ainda mais o teste e a constatação da aplicação prática do projeto.

Além disso, o projeto é viável financeiramente. Segundo a Prof$^a$. Dr$^a$. Ana Cristina Kochem Vendramin, orientadora deste projeto, não será necessária a compra de equipamentos e, uma vez que o projeto envolverá apenas \textit{software}, com a utilização de ferramentas gratuitas ou livres -- conforme já exposto neste documento -- os custos financeiros serão mínimos ou nulos.

%Relate a viabilidade técnica e financeira do projeto.
\newpage

\section{Cronograma Preliminar}\label{s:cronograma}

A Tabela \ref{tab:cronograma} fornece informações acerca das etapas necessárias para o desenvolvimento do projeto, bem como os seus tempos de duração estimados.

\begin{table}[h!]
%Se o nome da tabela for muito grande, é possível colocar entre colchetes o título reduzido que será mostrado na lista de tabelas e entre aspas o título completo que será mostrado acima da tabela. O mesmo é válido para figuras.

\caption[Cronograma preliminar]{Cronograma preliminar. Fonte: Autoria própria.}
\begin{center}
%\scalebox{0.75}{
\begin{tabular}{|>{\centering\arraybackslash}p{2cm}|>{\centering\arraybackslash}p{8.5cm}|>{\centering\arraybackslash}p{2cm}|}
%{|p{2.75cm}|p{7.75cm}|p{2.75cm}|}
\hline
\textbf{Número}     &\textbf{Etapa}      & \textbf{Dias} \\ \hline \hline
1 & Estudo de APIs e do Android SDK & 10 \\ \hline
2 & Projeto de software (modelagem com UML) & 10 \\ \hline
3 & Implementação do projeto de software & 30 \\ \hline
4 & Testes do protótipo utilizando um dispositivo móvel -- emulado -- com sistema operacional Android & 7 \\ \hline
5 & Testes do protótipo utilizando um dispositivo móvel -- real -- com sistema operacional Android & 7 \\ \hline
6 & Finalização do artefato & 14\\ \hline
7 & Finalização da documentação & 14\\ \hline
 & \raggedleft \textbf{Total} & 92\\ \hline
\end{tabular}%}
\end{center}
\label{tab:cronograma}
\end{table}