\chapter{Conclusão}\label{cap:conclusao}

Sabe-se que transporte coletivo é um tema em voga na maioria das cidades brasileiras, e também é uma área na qual aloca-se grande quantidade de recursos. Em especial, na cidade de Curitiba, tida por muitos anos como possuidora do melhor transporte público do país, há uma preocupação grande nesse aspecto. 

Ao longo do ano é possível visualizar por toda cidade, campanhas publicitárias voltadas para esse propósito, tendo em vista conscientizar o usuário do transporte coletivo acerca de alguns comportamentos, e também, tentando melhorar a qualidade do serviço prestado. Verifica-se também  que as tentativas de melhora do transporte coletivo não partem apenas da prefeitura da cidade, como  documentou-se nesse exposto, um grupo de desenvolvedores buscou materializar um aplicativo que mostrasse, em tempo real, a localização de cada ônibus em uma(s) determinada(s) linha(s) de interesse. No entanto, a aplicação apresentava um funcionamento um tanto instável, já que dependia dos serviços de um servidor da URBS para seu funcionamento correto. 

Numa tratativa que visa resolver esse problema, elaborou-se um projeto com intuito de elaborar um aplicativo que funcione sob uma arquitetura \textit{ad-hoc} e utilize recursos disponíveis em um \textit{smartphone}, e portanto, presente no dia a dia da  maioria dos usuários de ônibus na cidade. Entretanto, na acepção da proposta de projeto, encontrou-se certa dificuldade em definir o que o aplicativo ia fazer, quais seriam os recursos existentes no programa e principalmente, a forma de propagar a informação que faria o sistema, como um todo, funcionar. 

A escolha para a arquitetura de rede foi pensada de modo a aliar conceitos de DTN e P2P. A priori, a ideia apresentou-se confusa, já que as duas abstrações parecem paradoxais, todavia, existem registros na literatura de algoritmos que funcionam com essas características. Passados esses questionamentos iniciais o projeto progrediu de forma tranquila. A equipe trabalhou em conjunto quase na totalidade do tempo, o que facilitou a tomada de decisões e resolução de conflitos. 

No que tange projetos futuros, um enfoque interessante que poderia ser discutido apoia-se num mecanismo que garanta algum tipo de redundância na rede de comunicação. De certa forma, isso suportaria um prisma egrégio do projeto que é a atualização das informações em tempo real, ou seja, se uma abordagem falhar, existiria outra para garantir a propagação dos dados em um tempo exequível. Outra singularidade consiste em tornar acessíveis aos usuários, todas as linhas de ônibus cadastradas na URBS, sejam elas metropolitanas ou urbanas, de forma que haja cobertura total de informações em toda a cidade.

Por fim, busca-se com esse projeto fornecer um mecanismo de fácil manipulação e que traga benefícios à população em geral.